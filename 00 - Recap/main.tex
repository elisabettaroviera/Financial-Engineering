\documentclass[a4paper,12pt]{article}

% Preambolo
\usepackage[utf8]{inputenc}  % Supporto per caratteri UTF-8
\usepackage[margin=1cm,includefoot]{geometry}  % Margini ridotti (1 cm su tutti i lati)
\usepackage{titlesec}  % Personalizzazione dei titoli
\usepackage{setspace}  % Controllo della spaziatura
%\usepackage{parskip}   % Evita indentazioni, aggiunge spazio tra i paragrafi
\usepackage{enumitem}  % Per personalizzare gli elenchi
\usepackage{amssymb}
\usepackage{fancyhdr}
\usepackage{amsmath,amssymb}
\usepackage{amsmath}
\usepackage{graphicx}
\usepackage{subcaption}

% Spaziatura paragrafi
\setlength{\parskip}{2pt}
\setlength{\parindent}{0pt}

% Paragrafi compatti
\titlespacing*{\paragraph}{0pt}{4pt}{6pt}

% Liste senza spaziatura extra
\setlist{
  topsep=0pt,
  partopsep=0pt,
  itemsep=0pt,
  parsep=0pt
}




% --- LISTINGS (MATLAB look) ---
\usepackage[T1]{fontenc}
\usepackage[dvipsnames]{xcolor}
\usepackage{listings}

% Colori simili a MATLAB editor
\definecolor{MatlabBlue}{RGB}{0,0,255}
\definecolor{MatlabGreen}{RGB}{0,153,0}
\definecolor{MatlabPurple}{RGB}{153,0,153}
\definecolor{MatlabGray}{RGB}{120,120,120}

\lstdefinestyle{matlabStyle}{
  language=Matlab,
  basicstyle=\ttfamily\small,
  keywordstyle=\color{MatlabBlue},
  commentstyle=\color{MatlabGreen},
  stringstyle=\color{MatlabPurple},
  numbers=none,              % come MATLAB: niente numeri
  showstringspaces=false,
  breaklines=true,
  frame=single,              % riquadro
  rulecolor=\color{MatlabGray},
  tabsize=4,
  columns=fullflexible,
  keepspaces=true,
  % --- accenti robusti dentro lstlisting ---
  literate=
    {à}{{\`a}}1 {è}{{\`e}}1 {é}{{\'e}}1 {ì}{{\`i}}1 {ò}{{\`o}}1 {ù}{{\`u}}1
    {À}{{\`A}}1 {È}{{\`E}}1 {É}{{\'E}}1 {Ì}{{\`I}}1 {Ò}{{\`O}}1 {Ù}{{\`U}}1
}




% Operatore "opt" (max o min a seconda del contesto)
\DeclareMathOperator*{\opt}{opt}

\usepackage{graphicx}
\usepackage{tikz}
\usepackage{algorithm}
\usepackage{algpseudocode}
\usepackage[dvipsnames,svgnames]{xcolor}

% ------------------------------
% INDICE: solo section, senza titolo
% ------------------------------
\setcounter{tocdepth}{1}      % mostra solo \section
\renewcommand{\contentsname}{} % rimuove "Contents"

% Personalizzazione del titolo in alto
\makeatletter
\renewcommand{\maketitle}{
    \begin{center}
        \vspace{-2cm}
        {\LARGE \textbf{\@title}} \\[-0.2cm]
    \end{center}
}


% Impostazioni per i numeri di pagina
\pagestyle{empty}


% Dettagli del documento
\title{Financial Engineering}
\date{}
\titlespacing*{\section}{0pt}{-20pt}{4pt}


\begin{document}

%\maketitle
%\tableofcontents

%\vspace{-5cm}
\textcolor{NavyBlue}{\section{Recap Lezione 1 -- Financial Instruments and Markets}}

\paragraph{Idea generale della lezione}
La lezione introduce i \textbf{financial instruments} e i \textbf{financial markets}, presentando le principali categorie di \textbf{securities e contratti}, con particolare attenzione a \textbf{bonds}, \textbf{stocks}, \textbf{derivatives}, \textbf{forwards}, \textbf{futures} e \textbf{options}.  
L’obiettivo è fornire le definizioni di base, chiarire le differenze tra strumenti e introdurre i concetti di \textbf{payoff}, \textbf{price}, \textbf{risk} e \textbf{posizioni long/short}.

\paragraph{Securities e contratti}
Una \textbf{security} è un documento che conferisce diritti di proprietà su un \textbf{financial claim}.  
I mercati finanziari vengono classificati in base alle caratteristiche degli strumenti dal punto di vista dell’investitore.

\paragraph{Classificazione principale}
\begin{itemize}[label=-]
\item \textbf{Basic securities}
  \begin{itemize}[label=-]
  \item \textbf{Stocks} Securities che rappresentano una quota di proprietà di un’impresa
  \item \textbf{Fixed income (bonds)} Securities di debito che prevedono pagamenti determinati o predeterminabili
  \item \textbf{Foreign exchange (FX)} Strumenti legati allo scambio tra valute diverse
  \item \textbf{Commodities} Beni fisici standardizzati scambiati nei mercati finanziari
  \end{itemize}
\item \textbf{Derivatives and contracts}
  \begin{itemize}[label=-]
  \item \textbf{Forwards e futures} Contratti che fissano oggi le condizioni di uno scambio futuro
  \item \textbf{Options (call e put)} Contratti che conferiscono il diritto, ma non l’obbligo, di comprare o vendere un underlying
  \end{itemize}
\end{itemize}

\paragraph{Bonds}
Un \textbf{bond} è una security che rappresenta uno strumento di debito e conferisce al detentore il diritto a ricevere flussi di cassa futuri determinati.

\begin{itemize}[label=-]
\item Un pagamento predeterminato a una data futura detta \textbf{maturity}
\item Un \textbf{nominal value} detto anche face value, par value o principal
\end{itemize}

Caratteristiche principali
\begin{itemize}[label=-]
\item Il \textbf{bond price} è il prezzo pagato dal creditore al debitore
\item Il \textbf{nominal value} è l’importo rimborsato alla maturity
\item L’\textbf{interest rate} è espresso come percentuale del nominal value
\item I bonds sono soggetti a \textbf{default risk}
\end{itemize}

\paragraph{Tipologie di bonds}
\begin{itemize}[label=-]
\item \textbf{Maturity}
  \begin{itemize}[label=-]
  \item Short-term bond con maturity minore o uguale a un anno
  \item Long-term bond con maturity superiore a un anno
  \end{itemize}
\item \textbf{Pure discount bond / zero coupon bond}
  \begin{itemize}[label=-]
  \item Bond che prevede un unico pagamento alla maturity
  \item Prezzo iniziale inferiore al nominal value
  \end{itemize}
\item \textbf{Coupon bond}
  \begin{itemize}[label=-]
  \item Bond che prevede pagamenti periodici detti \textbf{coupons}
  \item Alla maturity pagamento dell’ultimo coupon più il nominal value
  \end{itemize}
\end{itemize}

\paragraph{Stocks}
Una \textbf{stock} è una security che conferisce al detentore una quota di proprietà dell’impresa emittente.

\begin{itemize}[label=-]
\item I profitti possono essere reinvestiti oppure distribuiti come \textbf{dividends}
\item I dividends non sono garantiti
\item Il prezzo è noto al tempo iniziale ma non nel futuro
\end{itemize}

\paragraph{Risk}
L’acquisto e la successiva vendita di una stock può generare un risultato incerto.

\begin{itemize}[label=-]
\item \textbf{Profit} Quando il return è positivo
\item \textbf{Loss} Quando il return è negativo
\end{itemize}

Il \textbf{return} dipende dalla differenza tra prezzo di vendita e prezzo di acquisto.  
La \textbf{capital gain} è la differenza tra selling price e initial price, mentre il rapporto tra dividend e prezzo è detto \textbf{dividend yield}.

\paragraph{Posizioni}
\begin{itemize}[label=-]
\item \textbf{Long position} Posizione che consiste nell’acquisto di una security
\item \textbf{Short position} Posizione che consiste nella vendita di una security non posseduta
\item Il venditore short beneficia di una riduzione del prezzo
\item Ogni contratto è uno \textbf{zero-sum game}
\end{itemize}

\paragraph{Derivatives}
Un \textbf{derivative} è uno strumento finanziario il cui valore dipende dal valore di un \textbf{underlying asset}.  
I derivatives sono anche detti \textbf{contingent claims} e permettono il trasferimento del rischio tra operatori.

\paragraph{Forwards e Futures}
Un \textbf{forward contract} è un accordo in cui le parti fissano oggi prezzo, quantità e data futura di consegna, con regolamento alla maturity e tipica negoziazione \textbf{OTC}.  
Un \textbf{future contract} è un contratto simile ma scambiato in un \textbf{exchange} ed è soggetto a \textbf{marking to market}.

\paragraph{Spot price, forward price e payoff}
Lo \textbf{spot price} è il prezzo corrente dell’underlying.  
Il \textbf{forward price} è il prezzo fissato oggi per uno scambio futuro.

\begin{itemize}[label=-]
\item Payoff long position pari a $S_T - K$
\item Payoff short position pari a $K - S_T$
\end{itemize}

Il forward contract è uno \textbf{zero-sum game}.

\paragraph{Options}
Una \textbf{option} è una security che conferisce al detentore il diritto, ma non l’obbligo, di comprare o vendere un underlying a un prezzo prefissato.

\paragraph{Tipologie di options}
\begin{itemize}[label=-]
\item \textbf{Call option} Diritto di acquistare l’underlying
\item \textbf{Put option} Diritto di vendere l’underlying
\end{itemize}

\paragraph{Payoff delle options}
\begin{itemize}[label=-]
\item Call long con payoff $(S_T - K)^+$
\item Put long con payoff $(K - S_T)^+$
\end{itemize}

Il prezzo dell’opzione è detto \textbf{option price} ed è pagato dal buyer al writer.

\newpage
\textcolor{RoyalBlue}{\section{Recap Lezione 2 -- Time Value of Money and Bonds}}
\paragraph{Idea generale della lezione}
La lezione formalizza il concetto di \textbf{time value of money} e introduce gli strumenti base del \textbf{money market}, con focus su \textbf{zero-coupon bonds} e \textbf{coupon bonds}.  
L’obiettivo è capire come confrontare valori monetari in tempi diversi tramite \textbf{capitalizzazione} e \textbf{sconto}, come definire correttamente un \textbf{return}, e come prezzare e confrontare investimenti tramite \textbf{discount factors}, \textbf{zero rates}, \textbf{yield} e \textbf{par yield}.

\paragraph{Time value of money}
Si investe oggi un capitale (principal) \(P\) e si vuole determinare il valore futuro \(A(t)\) al tempo \(t\), oppure viceversa si conosce un valore futuro e si vuole trovare il valore equivalente oggi.  
L’interesse maturato è \(I(t)=A(t)-P\). Per unità di tempo \(t=1\), si ha \(A-P=I\).  
Il \textbf{tasso di interesse} è definito come
\[
r=\frac{I}{P}
\qquad\Rightarrow\qquad
I=Pr.
\]

\paragraph{Simple compounding}
L’interesse cresce in modo \textbf{lineare} nel tempo, senza ``interesse sull’interesse''.  
L’ammontare al tempo \(t\) è
\[
A(t)=P+Prt=P(1+rt),
\]
con \textbf{growth factor} \(1+rt\).  
Lo sconto corrispondente è
\[
P=\frac{A(t)}{1+rt}.
\]
La coerenza delle unità di tempo è essenziale, il tasso deve essere espresso nella stessa unità del tempo utilizzato.

\paragraph{Compound compounding}
L’interesse viene \textbf{capitalizzato periodicamente}, quindi ogni periodo l’interesse maturato entra nel capitale e genera ulteriore interesse.  
Con \(r\) \textbf{nominal annual rate} convertibile \(m\) volte l’anno, il tasso per periodo è \(\frac{r}{m}\) e dopo \(t\) anni
\[
A(t)=P\left(1+\frac{r}{m}\right)^{mt}.
\]
Il valore attuale si ottiene con lo sconto
\[
P=\frac{A(t)}{\left(1+\frac{r}{m}\right)^{mt}}.
\]

\paragraph{Nominal annual rate}
Si introduce un tasso annuo \textbf{nominale} \(r\) quando la capitalizzazione avviene \(m\) volte l’anno, in modo che il tasso per periodo sia \(\frac{r}{m}\) e l’accumulazione su \(t\) anni sia coerente con la convenzione di capitalizzazione periodica
\[
A(t)=P\left(1+\frac{r}{m}\right)^{mt}.
\]

\paragraph{Effective annual rate}
Si vogliono confrontare investimenti con diverse frequenze di capitalizzazione tramite un unico tasso annuo ``comparabile''.  
L’\textbf{effective annual rate} \(r_e\) è definito imponendo equivalenza a un anno
\[
P(1+r_e)=P\left(1+\frac{r}{m}\right)^m
\qquad\Rightarrow\qquad
r_e=\left(1+\frac{r}{m}\right)^m-1.
\]

\paragraph{Continuous compounding}
La capitalizzazione avviene in modo ``continuo'' nel tempo.  
Passando al limite \(m\to\infty\)
\[
A(t)=Pe^{rt},
\qquad
P=A(t)e^{-rt}.
\]
La relazione con l’effective annual rate è
\[
Pe^r=P(1+r_e)
\qquad\Rightarrow\qquad
r=\ln(1+r_e),
\quad
r_e=e^r-1.
\]

\paragraph{Equation of value}
Si hanno più flussi di cassa in tempi diversi (prestiti, rimborsi, pagamenti) e si vuole imporre che siano \textbf{equivalenti} rispetto a un tempo di riferimento (tipicamente \(t=0\)).  
Il principio è riportare tutti i flussi allo stesso tempo con i fattori di sconto e imporre l’uguaglianza tra valori attuali. In forma generale
\[
\sum_{i} \text{CF}_i \cdot \text{DF}(t_i)=0,
\]
dove \(\text{DF}(t_i)\) è il discount factor coerente con la convenzione di capitalizzazione adottata.

\paragraph{Return: linear return e log-return}
Si osserva un valore \(V(s)\) al tempo \(s\) e un valore \(V(t)\) al tempo \(t\), e si vuole misurare la performance dell’investimento tra \(s\) e \(t\).  
Il \textbf{linear return} è
\[
LR(s,t)=\frac{V(t)-V(s)}{V(s)}.
\]
Nel caso di interesse semplice, il return è coerente con l’additività su intervalli consecutivi. Con capitalizzazione non lineare (es.\ continua) tale additività non vale in generale.  
Si introduce il \textbf{log-return}
\[
R(s,t)=\ln\!\left(\frac{V(t)}{V(s)}\right),
\]
che soddisfa sempre l’additività
\[
R(t_1,t_2)+R(t_2,t_3)=R(t_1,t_3).
\]

\paragraph{Money market}
Si considerano strumenti \textbf{risk-free} che generano flussi deterministici e che vengono usati come ``mattoni'' per prezzare cash flow futuri.  
Tra gli strumenti si considerano \textbf{zero coupon bonds}, \textbf{coupon bonds}, \textbf{bond forwards}, \textbf{annuities} e \textbf{bank account}.  
Il formalismo chiave è quello dei \textbf{discount factors} e dei \textbf{zero rates}.

\paragraph{Bank account}
Si considera un conto bancario come strumento risk-free che permette di accumulare capitale nel tempo secondo una convenzione di capitalizzazione scelta.  
Indicando con \(A(t)\) l’ammontare al tempo \(t\), le forme tipiche sono \(A(t)=P(1+rt)\), \(A(t)=P\left(1+\frac{r}{m}\right)^{mt}\) oppure \(A(t)=Pe^{rt}\), con attualizzazione ottenuta invertendo la rispettiva relazione.

\paragraph{Annuity}
Si considera un’\textbf{annuity} come una sequenza di pagamenti deterministici a date prefissate.  
Il valore oggi è la somma dei valori attuali dei pagamenti, ottenuti scontando ciascun flusso con il fattore di sconto coerente con la struttura dei tassi.

\paragraph{Investment in bond}
Si interpreta l’acquisto di un bond come un investimento che scambia un esborso certo oggi con una sequenza di flussi futuri deterministici.  
Il confronto tra investimenti in bond passa attraverso attualizzazione dei flussi, uso di zero rates e sintesi tramite yield e par yield.

\paragraph{Zero-coupon bonds (ZCB)}
Si vuole prezzare oggi un titolo che paga \textbf{un solo} flusso certo a scadenza \(T\).  
Indicando con \(B(t,T)\) il valore al tempo \(t<T\)
\[
B(T,T)=1.
\]
Se il bond è scambiato a mercato, lo \textbf{zero rate} implicito (continuamente composto) è definito da
\[
B(0,T)=e^{-r(0,T)T}
\qquad\Rightarrow\qquad
r(0,T)=-\ln\big(\frac{B(0,T)}{T}\big),
\]
e analogamente
\[
B(t,T)=e^{-r(t,T)(T-t)}
\qquad\Rightarrow\qquad
r(t,T)=-\ln\big(\frac{B(t,T)}{T-t}\big).
\]

\paragraph{Coupon bonds}
Si vuole prezzare oggi un titolo che paga cedole a date intermedie e rimborsa il face value alla maturity.  
Il prezzo teorico è il valore attuale di tutti i flussi. Se le cedole \(c_i\) sono pagate ai tempi \(t_i\) e \(t_n=T\), con face value \(F=1\)
\[
C(0,T)=\sum_{i=1}^{n-1} c_i\,e^{-r_i t_i}+(c_n+1)e^{-r_n t_n},
\]
dove \(r_i\) sono i tassi (zero rates) associati alle diverse scadenze.

\paragraph{Bond yield}
Si osserva un prezzo di mercato di un bond e si vuole riassumere l’intera struttura dei flussi con un \textbf{unico} tasso equivalente.  
Lo \textbf{yield} \(y\) è il tasso (unico) che, applicato a tutti i flussi del bond, riproduce il prezzo
\[
P=\sum_{i=1}^{n} \text{CF}_i\,e^{-y t_i},
\]
dove \(\text{CF}_i\) sono i flussi di cassa (cedole e rimborso) ai tempi \(t_i\).

\paragraph{Par yield}
Si vuole determinare quale tasso cedolare rende un bond scambiato ``alla pari'', cioè con prezzo uguale al face value.  
Il \textbf{par yield} è il coupon rate \(c\) tale che il prezzo del bond soddisfi
\[
P=F,
\]
con \(P\) dato dalla somma dei valori attuali dei flussi (cedole costruite con \(c\) e rimborso del face value) scontati tramite la struttura dei tassi.

\end{document}
