\documentclass[a4paper,12pt]{article}

% Preambolo
\usepackage[utf8]{inputenc}  % Supporto per caratteri UTF-8
\usepackage[margin=1cm,includefoot]{geometry}  % Margini ridotti (1 cm su tutti i lati)
\usepackage{titlesec}  % Personalizzazione dei titoli
\usepackage{setspace}  % Controllo della spaziatura
%\usepackage{parskip}   % Evita indentazioni, aggiunge spazio tra i paragrafi
\usepackage{enumitem}  % Per personalizzare gli elenchi
\usepackage{amssymb}
\usepackage{fancyhdr}
\usepackage{amsmath,amssymb}
\usepackage{amsmath}
\usepackage{graphicx}
\usepackage{subcaption}

% Spaziatura paragrafi
\setlength{\parskip}{2pt}
\setlength{\parindent}{0pt}

% Paragrafi compatti
\titlespacing*{\paragraph}{0pt}{4pt}{6pt}

% Liste senza spaziatura extra
\setlist{
  topsep=0pt,
  partopsep=0pt,
  itemsep=0pt,
  parsep=0pt
}




% --- LISTINGS (MATLAB look) ---
\usepackage[T1]{fontenc}
\usepackage[dvipsnames]{xcolor}
\usepackage{listings}

% Colori simili a MATLAB editor
\definecolor{MatlabBlue}{RGB}{0,0,255}
\definecolor{MatlabGreen}{RGB}{0,153,0}
\definecolor{MatlabPurple}{RGB}{153,0,153}
\definecolor{MatlabGray}{RGB}{120,120,120}
\definecolor{SoftGold}{RGB}{184,134,11} % Goldenrod
\definecolor{DarkGold}{RGB}{150,110,0}

\lstdefinestyle{matlabStyle}{
  language=Matlab,
  basicstyle=\ttfamily\small,
  keywordstyle=\color{MatlabBlue},
  commentstyle=\color{MatlabGreen},
  stringstyle=\color{MatlabPurple},
  numbers=none,              % come MATLAB: niente numeri
  showstringspaces=false,
  breaklines=true,
  frame=single,              % riquadro
  rulecolor=\color{MatlabGray},
  tabsize=4,
  columns=fullflexible,
  keepspaces=true,
  % --- accenti robusti dentro lstlisting ---
  literate=
    {à}{{\`a}}1 {è}{{\`e}}1 {é}{{\'e}}1 {ì}{{\`i}}1 {ò}{{\`o}}1 {ù}{{\`u}}1
    {À}{{\`A}}1 {È}{{\`E}}1 {É}{{\'E}}1 {Ì}{{\`I}}1 {Ò}{{\`O}}1 {Ù}{{\`U}}1
}




% Operatore "opt" (max o min a seconda del contesto)
\DeclareMathOperator*{\opt}{opt}

\usepackage{graphicx}
\usepackage{tikz}
\usepackage{algorithm}
\usepackage{algpseudocode}
\usepackage[dvipsnames,svgnames]{xcolor}

% ------------------------------
% INDICE: solo section, senza titolo
% ------------------------------
\setcounter{tocdepth}{1}      % mostra solo \section
\renewcommand{\contentsname}{} % rimuove "Contents"

% Personalizzazione del titolo in alto
\makeatletter
\renewcommand{\maketitle}{
    \begin{center}
        \vspace{-2cm}
        {\LARGE \textbf{\@title}} \\[-0.2cm]
    \end{center}
}


% Impostazioni per i numeri di pagina
\pagestyle{fancy}
\fancyhf{}
\fancyfoot[R]{\thepage}
\renewcommand{\headrulewidth}{0pt}
\renewcommand{\footrulewidth}{0pt}


% Dettagli del documento
\title{Financial Engineering}
\date{}
\titlespacing*{\section}{0pt}{-20pt}{4pt}
% BLU → AZZURRO
\definecolor{BlueDeep}{RGB}{28,58,117}
\definecolor{BlueRoyal}{RGB}{41,98,166}
\definecolor{AzureSoft}{RGB}{70,140,200}
\definecolor{CyanLight}{RGB}{102,180,210}

% AZZURRO → VERDE
\definecolor{AquaPastel}{RGB}{120,200,190}
\definecolor{MintGreen}{RGB}{120,200,150}
\definecolor{GreenSoft}{RGB}{90,170,120}
\definecolor{GreenLeaf}{RGB}{60,140,90}

% VERDE → GIALLO
\definecolor{OliveSoft}{RGB}{130,160,80}
\definecolor{LimePastel}{RGB}{180,200,90}
\definecolor{YellowSoft}{RGB}{240,220,120}

% GIALLO → ARANCIONE
\definecolor{GoldenSoft}{RGB}{240,190,90}
\definecolor{OrangeSoft}{RGB}{240,160,90}
\definecolor{OrangeWarm}{RGB}{230,130,70}

% ARANCIONE → ROSSO
\definecolor{RedSoft}{RGB}{220,100,80}
\definecolor{RedMuted}{RGB}{200,70,70}

% ROSSO → ROSA
\definecolor{RoseSoft}{RGB}{210,120,150}
\definecolor{PinkPastel}{RGB}{220,150,180}

% ROSA → VIOLA
\definecolor{LilacSoft}{RGB}{180,140,200}
\definecolor{PurpleSoft}{RGB}{140,110,180}



\begin{document}

\maketitle
%\tableofcontents

\vspace{-0.2cm}
\textcolor{BlueDeep}{\section{Financial Instruments and Markets}}

\paragraph{RECAP}
La lezione introduce i \textbf{financial instruments} e i \textbf{financial markets}, presentando le principali categorie di \textbf{securities e contratti}, con particolare attenzione a \textbf{bonds}, \textbf{stocks}, \textbf{derivatives}, \textbf{forwards}, \textbf{futures} e \textbf{options}.  
L’obiettivo è fornire le definizioni di base, chiarire le differenze tra strumenti e introdurre i concetti di \textbf{payoff}, \textbf{price}, \textbf{risk} e \textbf{posizioni long/short}.

\paragraph{Securities e contratti}
Una \textbf{security} è un documento che conferisce diritti di proprietà su un \textbf{financial claim}.  
I mercati finanziari vengono classificati in base alle caratteristiche degli strumenti dal punto di vista dell’investitore.

\textbf{Classificazione principale}
\begin{itemize}[label=-]
\item \textbf{Basic securities}
  \begin{itemize}[label=-]
  \item \textbf{Stocks} Securities che rappresentano una quota di proprietà di un’impresa
  \item \textbf{Bonds} Securities di debito che prevedono pagamenti determinati o predeterminabili
  \item \textbf{Foreign exchange} Strumenti legati allo scambio tra valute diverse
  \item \textbf{Commodities} Beni fisici standardizzati scambiati nei mercati finanziari
  \end{itemize}
\item \textbf{Derivatives and contracts}
  \begin{itemize}[label=-]
  \item \textbf{Forwards e futures} Contratti che fissano oggi le condizioni di uno scambio futuro
  \item \textbf{Options} Contratti che danno il diritto, ma non l’obbligo, di comprare o vendere un underlying
  \end{itemize}
\end{itemize}

\paragraph{Bonds}
Un \textbf{bond} è una security che rappresenta uno strumento di debito e conferisce al detentore il diritto a ricevere flussi di cassa futuri determinati $\rightarrow$ pagamento predeterminato a una data futura (\textbf{maturity}).

\begin{itemize}[label=-]
\item Il \textbf{bond price} è il prezzo pagato dal creditore al debitore
\item Il \textbf{nominal value} è l’importo rimborsato alla maturity
\item L’\textbf{interest rate} è espresso come percentuale del nominal value
\item I bonds sono soggetti a \textbf{default risk}
\end{itemize}

\textbf{Tipologie di bonds}
\begin{itemize}[label=-]
\item \textbf{Maturity}
  \begin{itemize}[label=-]
  \item \textbf{Short-term} bond con maturity minore o uguale a un anno
  \item \textbf{Long-term} bond con maturity superiore a un anno
  \end{itemize}
\item \textbf{Pure discount bond / zero coupon bond}
  \begin{itemize}[label=-]
  \item Bond che prevede un unico pagamento alla maturity
  \item Prezzo iniziale inferiore al nominal value
  \end{itemize}
\item \textbf{Coupon bond}
  \begin{itemize}[label=-]
  \item Bond che prevede pagamenti periodici detti \textbf{coupons}
  \item Alla maturity pagamento dell’ultimo coupon più il nominal value
  \end{itemize}
\end{itemize}

\paragraph{Stocks}
Una \textbf{stock} è una security che conferisce al detentore una quota di proprietà dell’impresa emittente.

\begin{itemize}[label=-]
\item I profitti possono essere reinvestiti oppure distribuiti come \textbf{dividends}
\item I dividends non sono garantiti
\item Il prezzo è noto al tempo iniziale ma non nel futuro
\end{itemize}

L’acquisto e la successiva vendita di una stock può generare un risultato incerto.
Il \textbf{return} dipende dalla differenza tra prezzo di vendita e prezzo di acquisto.  
La \textbf{capital gain} è la differenza tra selling price e initial price, mentre il rapporto tra dividend e prezzo è detto \textbf{dividend yield}.

\begin{itemize}[label=-]
\item \textbf{Profit} Quando il return è positivo
\item \textbf{Loss} Quando il return è negativo
\end{itemize}

\textbf{Posizioni}
\begin{itemize}[label=-]
\item \textbf{Long position} Posizione che consiste nell’acquisto (buy) di una security
\item \textbf{Short position} Posizione che consiste nella vendita (sell) di una security non posseduta
\item Il venditore short beneficia di una riduzione del prezzo
\item Ogni contratto è uno \textbf{zero-sum game}
\end{itemize}

\paragraph{Derivatives}
Un \textbf{derivative} è uno strumento finanziario il cui valore dipende dal valore di un \textbf{underlying asset}.  
I derivatives, \textbf{contingent claims}, permettono il trasferimento del rischio tra operatori.

\paragraph{Forwards e Futures}
Un \textbf{forward contract} è un accordo in cui le parti fissano oggi prezzo, quantità e data futura di consegna, con regolamento alla maturity e tipica negoziazione \textbf{OTC}.  
Un \textbf{future contract} è un contratto simile ma scambiato in un \textbf{exchange} ed è soggetto a \textbf{marking to market}.

\paragraph{Spot price, forward price e payoff}
Lo \textbf{spot price} è il prezzo corrente dell’underlying.  
Il \textbf{forward price} è il prezzo fissato oggi per uno scambio futuro. Il forward contract è uno \textbf{zero-sum game}.

\begin{itemize}[label=-]
\item \textbf{Payoff long position} pari a $S_T - K$
\item \textbf{Payoff short position} pari a $K - S_T$
\end{itemize}

\paragraph{Options}
Una \textbf{option} è una security che conferisce al detentore il diritto, ma non l’obbligo, di comprare o vendere un underlying a un prezzo prefissato.
Il prezzo dell’opzione è detto \textbf{option price} ed è pagato dal buyer al writer.

\textbf{Tipologie di options}
\begin{itemize}[label=-]
\item \textbf{Call option} Diritto di acquistare (buy) l’underlying
\item \textbf{Put option} Diritto di vendere (sell) l’underlying
\end{itemize}

\textbf{Payoff delle options}
\begin{itemize}[label=-]
\item Call long con payoff $(S_T - K)^+$
\item Put long con payoff $(K - S_T)^+$
\end{itemize}

\vspace{0.5cm}

\textcolor{BlueRoyal}{\section{Time Value of Money and Bonds}}
\paragraph{RECAP}
La lezione formalizza il concetto di \textbf{time value of money} e introduce gli strumenti del \textbf{money market}, con focus su \textbf{zero-coupon bonds} e \textbf{coupon bonds}.  
L’obiettivo è capire come confrontare valori monetari in tempi diversi tramite \textbf{capitalizzazione} e \textbf{sconto}, come definire correttamente un \textbf{return}, e come prezzare e confrontare investimenti tramite \textbf{discount factors}, \textbf{zero rates}, \textbf{yield} e \textbf{par yield}.

\paragraph{Simple capitalization}
Si investe oggi un capitale \(P\) e si vuole determinare il valore futuro \(A(t)\) al tempo \(t\), oppure viceversa si conosce un valore futuro e si vuole trovare il valore equivalente oggi.  
L’interesse maturato è \(I(t)=A(t)-P\). Per unità di tempo \(t=1\), si ha \(A-P=I\).  
Il \textbf{tasso di interesse} è definito come
\(
I=Pr.
\)
L’interesse cresce in modo \textbf{lineare} nel tempo, senza ``interesse sull’interesse''.  
L’ammontare al tempo \(t\) è$$A(t)=P(1+rt).$$
La coerenza delle unità di tempo è essenziale, il tasso deve essere espresso nella stessa unità del tempo utilizzato.

\paragraph{Compounding}
L’interesse viene \textbf{capitalizzato periodicamente}, quindi ogni periodo l’interesse maturato entra nel capitale e genera ulteriore interesse.  
Con \(r\) \textbf{nominal annual rate} convertibile \(m\) volte l’anno, il tasso per periodo è \(\frac{r}{m}\) e dopo \(t\) anni
\[
A(t)=P\left(1+\frac{r}{m}\right)^{mt}.
\]

\paragraph{Effective annual rate}
Si vogliono confrontare investimenti con diverse frequenze di capitalizzazione tramite un unico tasso annuo ``comparabile''.  
L’\textbf{effective annual rate} \(r_e\) è definito imponendo equivalenza a un anno
\[
P(1+r_e)=P\left(1+\frac{r}{m}\right)^m
\qquad\Rightarrow\qquad
r_e=\left(1+\frac{r}{m}\right)^m-1.
\]

\paragraph{Continuous compounding}
La capitalizzazione avviene in modo ``continuo'' nel tempo.  
Passando al limite \(m\to\infty\)
\[
A(t)=Pe^{rt},
\qquad
P=A(t)e^{-rt}.
\]
La relazione con l’effective annual rate è
\[
Pe^r=P(1+r_e)
\qquad\Rightarrow\qquad
r=\ln(1+r_e),
\quad
r_e=e^r-1.
\]

\paragraph{Equation of value}
Si hanno più flussi di cassa in tempi diversi (prestiti, rimborsi, pagamenti) e si vuole imporre che siano \textbf{equivalenti} rispetto a un tempo di riferimento (tipicamente \(t=0\)).  
Il principio è riportare tutti i flussi allo stesso tempo con i fattori di sconto e imporre l’uguaglianza tra valori attuali. 

\paragraph{Return: linear return e log-return}
Si osserva un valore \(V(s)\) al tempo \(s\) e un valore \(V(t)\) al tempo \(t\), e si vuole misurare la performance dell’investimento tra \(s\) e \(t\).  
Il \textbf{linear return} è
\[
LR(s,t)=\frac{V(t)-V(s)}{V(s)} \qquad\Rightarrow\qquad LR(s,t)=r(t-s).
\]
Nel caso di interesse semplice, il return è coerente con l’additività su intervalli consecutivi. Con capitalizzazione non lineare (es.\ continua) tale additività non vale in generale.  
Si introduce il \textbf{log-return}
\[
R(s,t)=\ln\!\left(\frac{V(t)}{V(s)}\right) \;\;\Rightarrow\;\; \text{che soddisfa sempre l’additività}\;\; R(t_1,t_2)+R(t_2,t_3)=R(t_1,t_3) \quad\text{\textbf{PROP}}.
\]

\paragraph{Money market}
Si considerano strumenti \textbf{risk-free} che generano flussi deterministici e che vengono usati come ``mattoni'' per prezzare cash flow futuri.  
Tra gli strumenti si considerano \textbf{zero coupon bonds}, \textbf{coupon bonds}, \textbf{bond forwards}, \textbf{annuities} e \textbf{bank account}.  
Il formalismo chiave è quello dei \textbf{discount factors} e dei \textbf{zero rates}.

\emph{\textbf{Bank account}
Si considera un conto bancario come strumento risk-free che permette di accumulare capitale nel tempo secondo una convenzione di capitalizzazione scelta.  
Indicando con \(A(t)\) l’ammontare al tempo \(t\), le forme tipiche sono \(A(t)=P(1+rt)\), \(A(t)=P\left(1+\frac{r}{m}\right)^{mt}\) oppure \(A(t)=Pe^{rt}\), con attualizzazione ottenuta invertendo la rispettiva relazione.
}

\emph{\textbf{Investment in bond}
Si interpreta l’acquisto di un bond come un investimento che scambia un esborso certo oggi con una sequenza di flussi futuri deterministici.  
Il confronto tra investimenti in bond passa attraverso attualizzazione dei flussi, uso di zero rates e sintesi tramite yield e par yield.
}

\paragraph{1. Zero-coupon bonds (ZCB)}
Si vuole prezzare oggi un titolo che paga \textbf{un solo} flusso certo a scadenza \(T\). Mettere i soldi in un deposito è uguale a comprare con gli stessi un ZCB perchè è un investimento senza rischi.
Indicando con \(B(t,T)\) il valore al tempo \(t<T\)
\(\;
B(T,T)=1.
\)
Se il bond è scambiato a mercato, lo \textbf{zero rate} implicito (continuamente composto) è definito da
\[
\begin{aligned}
B(0,T) &= e^{-r(0,T)T}
&\Rightarrow\quad r(0,T) &= -\frac{1}{T}\ln\!\big(B(0,T)\big),
\\[0.6em]
B(t,T) &= e^{-r(t,T)(T-t)}
&\Rightarrow\quad r(t,T) &= -\frac{1}{T-t}\ln\!\big(B(t,T)\big).
\end{aligned}
\]

\paragraph{2. Coupon bonds}
Si vuole prezzare oggi un titolo che paga cedole a date intermedie e rimborsa il face value alla maturity.  
Il prezzo teorico è il valore attuale di tutti i flussi. Se le cedole \(c_i\) sono pagate ai tempi \(t_i\) e \(t_n=T\), 
\[
C(0,T)=\sum_{i=1}^{n-1} c_i\,e^{-r_i t_i}+(c_n+F)e^{-r_n t_n},
\]
dove \(r_i\) sono i tassi (zero rates) associati alle diverse scadenze.

\paragraph{3. Bond yield}
Si osserva un prezzo di mercato di un bond e si vuole riassumere l’intera struttura dei flussi con un \textbf{unico} tasso equivalente.  

\textbf{Par yield}
Si vuole determinare quale tasso cedolare rende un bond scambiato ``alla pari'', cioè con prezzo uguale al face value.  
Il \textbf{par yield} è il coupon rate \(c\) tale che il prezzo del bond soddisfi
\(
P=F,
\)
con \(P\) dato dalla somma dei valori attuali dei flussi (cedole costruite con \(c\) e rimborso del face value) scontati tramite la struttura dei tassi.

\textbf{4. Annuity}
Si considera un’\textbf{annuity} come una sequenza di pagamenti deterministici a date prefissate.  
Il valore oggi è la somma dei valori attuali dei pagamenti, ottenuti scontando ciascun flusso con il fattore di sconto coerente con la struttura dei tassi
$$PV=c\frac{1-(1+r)^n}{r}.$$

\textcolor{AzureSoft}{\section{Duration and Interest Rate Risk}}
\paragraph{RECAP}
La lezione analizza il \textbf{rischio di tasso di interesse} associato ai bond e introduce strumenti analitici per misurare la \textbf{sensibilità del prezzo di un titolo obbligazionario alle variazioni del rendimento}.  
Il concetto centrale è la \textbf{duration}, ottenuta a partire da un’approssimazione di primo ordine del prezzo del bond, e il suo utilizzo nella gestione del rischio e nella costruzione di strategie di copertura.

\paragraph{Bond price come funzione del rendimento}
Il prezzo di un bond può essere visto come una funzione del rendimento \(y\),
\[
B(0,T)=\sum_{i=1}^{n} c_i e^{-y t_i},
\]
dove \(c_i\) sono i flussi di cassa pagati ai tempi \(t_i\).

\paragraph{Sviluppo di Taylor del prezzo del bond}
Si considera lo sviluppo di Taylor del prezzo del bond attorno a un valore \(y_0\), dove \(D(y_0)\) e \(C(y_0)\) rappresentano rispettivamente duration e convexity valutate in \(y_0\).
\[
B(y)=B(y_0)-D(y_0)B(y_0)\Delta y+\frac{1}{2}C(y_0)B(y_0)(\Delta y)^2+\dots
\]

\paragraph{Duration}
La \textbf{duration} è definita come
\[
D(y) = -\frac{1}{B(y)}\frac{ dB(y)}{ dy}
   = \frac{\sum_{i=1}^{n} t_ic_i e^{-y t_i}}{B(y)}.
\]
La duration ammette le seguenti interpretazioni
\begin{itemize}[label=-]
\item È una \textbf{combinazione convessa dei tempi di pagamento}, poiché i pesi \(\frac{c_i e^{-y t_i}}{B(y)}\) sono positivi e sommano a uno
\item Misura la \textbf{sensitività del prezzo del bond} rispetto a variazioni del rendimento
\(
\frac{\Delta B}{B}\approx -D(y)\,\Delta y
\)
\item Costituisce una \textbf{misura di rischio di tasso di interesse}, quantificando l’esposizione del valore del bond a shock sui tassi
\end{itemize}

\paragraph{Convexity}
La \textbf{convexity} è definita come
\[
C(y) = \frac{1}{B(y)}\frac{d^2 B(y)}{dy^2}
  = \frac{\sum_{i=1}^{n} t_i^2 c_i e^{-y t_i}}{B(y)}.
\]

\paragraph{Portfolio of bonds}
Si considera un portafoglio $h=(h_1, \dots, h_n)$ di \(n\) bond con quantità \(h_i\) e prezzi \(B_i(y)\), per cui il valore del portafoglio è
\[
V_p(y)=\sum_{i=1}^{n} h_i B_i(y).
\]
Imponendo che la duration del portafoglio soddisfi
\[
-D_p(y)V_p(y)=\frac{dV_p(y)}{dy}=-\sum_{i=1}^{n} h_i D_i(y)B_i(y), \qquad
D_p(y)=\frac{\sum_{i=1}^{n} h_i D_i(y)B_i(y)}{V_p(y)}
      =\sum_{i=1}^{n} D_i(y)\bigg(\frac{h_i B_i(y)}{V_p(y)}\bigg).
\]
Si definiscono quindi
\[
w_i=\frac{h_i B_i(y)}{V_p(y)}, \qquad
D_p(y)=\sum_{i=1}^{n} w_i D_i(y).
\]

\paragraph{Dynamic hedging}
La duration fornisce una copertura valida localmente.  
Poiché il rendimento e il tempo influenzano il valore della duration, è necessario un aggiornamento continuo del portafoglio.  
Il \textbf{dynamic hedging} consiste nel ribilanciare dinamicamente le posizioni per mantenere sotto controllo l’esposizione al tasso di rischio.


\vspace{0.5cm}

\textcolor{CyanLight}{\section{Term Structure, Bootstrap and Forward Rates}}

\paragraph{RECAP}
La lezione studia la \textbf{struttura a termine dei tassi di interesse} senza imporre l’ipotesi che il rendimento sia indipendente dalla maturity.  
Si introducono gli \textbf{spot rates}, la costruzione della \textbf{zero curve tramite bootstrap}, le proprietà di non arbitraggio nel caso di struttura a termine deterministica e il concetto di \textbf{forward rate}, con applicazione ai \textbf{Forward Rate Agreements (FRA)}.

\paragraph{Struttura a termine e prezzo dei bond}
Si considera il prezzo di un zero-coupon bond al tempo \(t\) con maturity \(T\), indicato come \(B(t,T)\), dove \(T-t\) è il \textbf{time-to-maturity}.  
Senza assumere rendimenti costanti sulle maturity, il prezzo del bond è espresso come
\[
B(t,T)=e^{-y(t,T)(T-t)}.
\]

\paragraph{PROP.} Se il rendimento (yield) per qualche $t>0$ fosse noto già al tempo $t=0$, allora
\(
y(t)=y(0)\; \forall t,
\)
altrimenti sarebbe possibile costruire strategie di arbitraggio.

\paragraph{PROP.} Per un coupon bond, se $t \in (0,t_1] \; \Rightarrow \; D(t)=D(0)-t$, perchè la duration cambia con il tempo.
\textbf{NB} La duration cambia anche in funzione dello yield. La \textbf{convexity} può, quindi, essere utile per misurare quanto rapidamente la duration cambia in funzione dello yield.
\emph{
\paragraph{General term structure of zero-coupon bonds}
La struttura a termine dei zero-coupon bond è descritta dall’insieme dei prezzi \(B(t,T)\) al variare della maturity \(T\).  
Essa è equivalente alla struttura a termine dei rendimenti \(y(t,T)\), che sintetizzano l’informazione contenuta nei prezzi dei bond per ogni scadenza.
}
\emph{
\paragraph{Spot rate}
Fissando \(t=0\), il rendimento \(y(0,T)\) associato al bond \(B(0,T)\) è detto \textbf{spot rate}.  
Gli spot rates descrivono la struttura a termine iniziale dei tassi di interesse.}
\emph{
\paragraph{Prezzo di un coupon bond tramite spot rates}
Se \(y(0,T)\) dipende solo dalla maturity \(T\), il prezzo di un coupon bond può essere scritto come somma dei valori attuali dei flussi di cassa
\[
P = c_1 e^{-y(0,t_1)t_1} + c_2 e^{-y(0,t_2)t_2} + \dots + (c_n+F)e^{-y(0,T)T}.
\]}

\paragraph{Bootstrap method}
Definendo la funzione \(y(0,T)\) come \textbf{zero curve}, il \textbf{bootstrap method} viene utilizzato per costruirla a partire dai prezzi di mercato dei bond.  
Si procede in modo sequenziale
\begin{itemize}[label=-]
\item Si ricavano gli spot rates associati ai zero-coupon bond
\item Si utilizzano tali spot rates per scontare i flussi noti dei bond coupon
\item Si determina iterativamente il nuovo spot rate
\end{itemize}
Nel caso di coupon pagati semestralmente, ciascun flusso viene scontato con il tasso corrispondente alla sua maturity.

La \emph{zero curve} è la curva dei tassi spot dei titoli zero-coupon e rappresenta la struttura a termine fondamentale dei tassi di interesse. A ogni maturity $T$ associa un tasso $y(0,T)$ tale che il prezzo del corrispondente zero-coupon bond sia
\(
B(0,T)=e^{-y(0,T)\,T}.
\)


\paragraph{Arbitrage} L'\textbf{arbitrage} è una strategia di trading che partendo senza soldi assume probabilità zero di perderne e probabilità di uno di guadagnarne.

\paragraph{Argomento di non arbitraggio}
Nel mercato finanziario si assume che non ci siano possibilità di arbitraggio. La relazione viene dimostrata tramite la costruzione di un portafoglio che soddisfa le seguenti proprietà
\begin{itemize}[label=-]
\item Ha valore nullo a \(t=0\)
\item Ha valore nullo a \(t\)
\item Genera un payoff positivo certo a \(T\)
\end{itemize}

\paragraph{Term structure deterministica}
Se la struttura a termine \(y(t,T)\) è deterministica, l’assenza di arbitraggio implica la relazione
\[
B(0,T)=B(0,t)\,B(t,T).
\]
Se i tassi sono deterministici, la struttura a termine futura \(B(t,T)\) è completamente determinata dalla struttura iniziale.  
Vale inoltre la relazione
\[
B(t,T)=\frac{B(0,T)}{B(0,t)}=e^{y(0,t)t-y(0,T)T}.
\]
In generale, tuttavia, la struttura a termine futura non dipende interamente da quella iniziale.
\emph{
\paragraph{Forward rates deterministici}
Assumendo che
\(
B(0,T)=B(0,t)\,B(t,T),
\)
i forward rates \(y(t,T)\) risultano deterministici e dipendono solo dalla struttura iniziale.  
Si sottolinea esplicitamente che questa è un’assunzione utile, ma non necessariamente realistica.}
\emph{
\paragraph{Calcolo dei forward rates}
Assumendo continuous compounding, dalla relazione
\[
e^{-y(0,T)T}=e^{-y(0,t)t}\,e^{-y(t,T)(T-t)}, \qquad
y(t,T)=\frac{y(0,T)T-y(0,t)t}{T-t}.
\]
Conoscendo gli spot rates iniziali è quindi possibile determinare tutti i forward rates.}
\emph{
\paragraph{Forward Rate Agreement (FRA)}
Un \textbf{Forward Rate Agreement (FRA)} è un contratto derivato in cui le parti si scambiano, alla maturity del contratto, la differenza tra due tassi di interesse
\begin{itemize}[label=-]
\item Un tasso fisso detto forward rate
\item Un tasso variabile di mercato detto settlement rate
\end{itemize}
La differenza è moltiplicata per la durata del contratto e per il notional principal.}
\emph{
\paragraph{Interpretazione economica del FRA}
Le posizioni contrattuali nel FRA sono così definite
\begin{itemize}[label=-]
\item Il seller del FRA riceve il tasso fisso e paga il tasso variabile
\item Il buyer del FRA riceve il tasso variabile e paga il tasso fisso
\end{itemize}
Il FRA consente di coprirsi contro variazioni future dei tassi di interesse e permette di fissare oggi il rendimento di un investimento o di un prestito futuro.
}
\emph{
\paragraph{Determinazione anticipata dei forward rates}
Attraverso gli spot rates osservabili \(y(0,1)\) e \(y(0,2)\), è possibile costruire un’operazione equivalente tramite zero-coupon bond e determinare il forward rate \(y(1,2)\).  
Il forward rate rappresenta il tasso implicito che rende equivalente il valore attuale dei flussi futuri.}
\emph{
\paragraph{Caratteristiche del FRA}
Il FRA è un contratto con le seguenti caratteristiche
\begin{itemize}[label=-]
\item Non standardizzato
\item Negoziato over-the-counter (OTC)
\item Utilizzabile per hedging, speculazione e arbitraggio
\end{itemize}
}

\vspace{0.5cm}

\textcolor{AquaPastel}{\section{Forward and Futures Contracts}}

\paragraph{RECAP}
La lezione introduce in modo formale i \textbf{forward contracts} e i \textbf{futures contracts}, analizzandone \textbf{payoff}, \textbf{pricing per assenza di arbitraggio}, \textbf{valore nel tempo} e differenze strutturali.  
L’attenzione è posta sulla costruzione di strategie di arbitraggio e di \textbf{replicating portfolios}, sul ruolo dei \textbf{dividends}, del \textbf{dividend yield}, dei \textbf{storage costs} e sul funzionamento dei mercati futures con \textbf{marking to market} e \textbf{margin system}.

\paragraph{Forward contract}
Un \textbf{forward contract} è un accordo tra due parti per acquistare o vendere un asset a una data futura \(T\), detta \textbf{delivery time}, a un prezzo fissato oggi, detto \textbf{forward price}.  
Alla stipula non avviene alcun pagamento iniziale.

\paragraph{Forward price}
Il \textbf{forward price} \(F(0,T)\) è il prezzo che rende nullo il valore del forward contract al tempo iniziale.  
Esso è determinato in modo univoco dall’assenza di arbitraggio e dipende dal prezzo spot dell’asset, dal tasso di interesse e da eventuali flussi intermedi.
\emph{
\paragraph{Long e short forward position}
\begin{itemize}[label=-]
\item \textbf{Long forward position} Obbligo di acquistare l’asset a \(T\) al prezzo \(F(0,T)\)
\item \textbf{Short forward position} Obbligo di vendere l’asset a \(T\) al prezzo \(F(0,T)\)
\end{itemize}}
\emph{
\paragraph{Payoff del forward}
Alla data di consegna \(T\) il payoff è
\[
\text{Long}: S(T)-F(0,T),
\qquad
\text{Short}: F(0,T)-S(T).
\]
Il forward è uno \textbf{zero-sum game}.}

\paragraph{TEO. Pricing del forward su stock senza dividendi}
Si assume un tasso risk-free costante \(r\) con continuous compounding.  
Il forward price su uno stock che non paga dividendi è
\[
F(0,T)=S(0)e^{rT} \quad \Rightarrow \quad F(t,T)=S(t)e^{r(T-t)}.
\]
La relazione deriva dall’assenza di arbitraggio e dalla possibilità di replicare il payoff del forward tramite una strategia di trading dinamicamente coerente.

\paragraph{Argomento di non arbitraggio}
Se \(F(0,T)>S(0)e^{rT}\) oppure \(F(0,T)<S(0)e^{rT}\), è possibile costruire una strategia con valore nullo a \(t=0\) e payoff positivo certo a \(T\), generando arbitraggio.  
L’assenza di arbitraggio impone l’uguaglianza.

\paragraph{TEO. Pricing del forward su stock con dividendi discreti}
Se lo stock paga un dividendo discreto \(d_t\) a una data intermedia \(t\) con \(0<t<T\), il forward price è
\[
F(0,T)=\bigl(S(0)-e^{-rt}d_t\bigr)e^{rT}.
\]
La dimostrazione utilizza un portafoglio che replica il payoff del forward includendo l’incasso dei dividendi intermedi.

\paragraph{Dividend yield continuo}
In molti contesti i dividendi sono modellati come pagati continuamente a un tasso costante \(\delta\), detto \textbf{dividend yield}.  
Una unità di stock detenuta da \(0\) a \(T\) genera un flusso continuo proporzionale al prezzo dell’asset.

\paragraph{Forward su asset con dividend yield}
Per uno stock che paga dividendi continuamente a tasso \(\delta\), il forward price è
\[
F(0,T)=S(0)e^{(r-\delta)T}.
\]

\paragraph{Valore di un forward contract}
Alla stipula il valore del forward è nullo.  
Nel tempo, il valore può diventare positivo o negativo in funzione dell’evoluzione del forward price.

\paragraph{PROP. Valore del forward a una data intermedia}
Per \(0\le t\le T\), il valore al tempo \(t\) di un long forward stipulato a \(t=0\) è
\[
V(t)=\bigl[F(t,T)-F(0,T)\bigr]e^{-r(T-t)}.
\]

\paragraph{Caso specifico senza dividendi}
Se l’asset non paga dividendi
\(
V(t)=S(t)-S(0)e^{rt}.
\)

\paragraph{Futures contracts}
Un \textbf{futures contract} è simile a un forward ma è scambiato su un exchange ed è standardizzato.  
Il rischio di default è eliminato tramite il sistema di \textbf{marking to market} e di margini.

\textbf{Marking to market}
\begin{itemize}[label=-]
\item Regolazione quotidiana di profitti e perdite
\item Valore nullo della posizione futures a ogni istante
\end{itemize}

\textbf{Margin system}
\begin{itemize}[label=-]
\item \textbf{Initial margin} Deposito iniziale a garanzia
\item \textbf{Maintenance margin} Livello minimo del deposito
\item \textbf{Margin call} Richiesta di reintegro del margine
\end{itemize}
Il clearing house chiude la posizione se la margin call non viene soddisfatta, eliminando il rischio di default.

\paragraph{Liquidità nei futures markets}
Un’importante caratteristica dei futures markets è l’elevata \textbf{liquidity}, dovuta a
\begin{itemize}[label=-]
\item Standardizzazione dei contratti che rende gli strumenti omogenei e facilmente scambiabili
\item Presenza del clearing house che garantisce l’adempimento delle obbligazioni
\end{itemize}
\emph{
\paragraph{Pricing dei futures}
Se il tasso di interesse è costante
\[
f(0,T)=F(0,T),
\qquad
f(t,T)=S(t)e^{r(T-t)}.
\]
Se i tassi sono deterministici ma variabili nel tempo, la relazione rimane valida con opportune modifiche.}

\paragraph{Portfolio}
Un \textbf{portfolio} è un vettore \(h=(x,y)\) che rappresenta le quantità detenute di due asset.  
Il valore del portafoglio è
\[
V_h(t)=xV_X(t)+yV_Y(t).
\]

\paragraph{Replicating portfolio}
Un portafoglio è detto \textbf{replicating portfolio} se riproduce esattamente il payoff di un derivato a scadenza.

\paragraph{Portafoglio replicante di un forward}
Un possibile portafoglio replicante per un long forward è
\[
h=\bigl(-F(0,T)e^{-rT},\,1\bigr).
\]
Il valore del portafoglio è
\[
V_h(0)=-F(0,T)e^{-rT}+S(0),
\qquad
V_h(T)=S(T)-F(0,T).
\]
\emph{
\paragraph{Principio di no-arbitrage pricing}
Se un claim è replicabile tramite un portafoglio, allora il suo prezzo iniziale deve coincidere con il valore iniziale del portafoglio.  
Per un forward
\(
V_f(0)=0.
\)}

\textbf{NB} 
\begin{itemize}[label=-]
  \item \textbf{Arbitrage strategy}: cerca di costruire un arbitrage
  \item \textbf{Replicating strategy}: cerca di replicare un payoff di un derivative
  \item \textbf{Hedging strategy}: cerca di rimuovere un rischio
\end{itemize}

\paragraph{Consumption assets}
Un \textbf{consumption asset} è un bene fisico che deve essere stoccato.  
I forward e futures su tali asset sono rilevanti per imprese industriali.

\paragraph{Investment asset}
Se l’asset è un investment asset e non genera costi o benefici intermedi
\[
F(0,T)=S(0)e^{rT}.
\]

\paragraph{Storage costs}
I consumption assets comportano costi di stoccaggio che possono essere interpretati come flussi negativi durante la vita del contratto.  
Indicando con \(U\) il valore attuale dei storage costs.
Se l’asset è un consumption asset
\[
F(0,T)=(S(0)+U)e^{rT}.
\]

\vspace{0.5cm}

\textcolor{MintGreen}{\section{Hedging with Futures}}

\paragraph{RECAP}
La lezione introduce come coprire (hedging) un’esposizione al rischio di prezzo di un asset \(S(t)\) usando forward e soprattutto futures.  
Si evidenzia che il forward può dare copertura perfetta quando l’orizzonte di hedging coincide con la maturity, mentre con i futures in generale compare il \textbf{basis risk}, quindi l’hedging viene impostato come un problema di \textbf{minimizzazione della varianza}.

\paragraph{Scenario e motivazione}
Un modo semplice per coprire l’esposizione alle variazioni di \(S(t)\) è entrare in un forward, ma un forward può non essere facilmente disponibile e presenta rischio di controparte.  
Il mercato futures è invece più sviluppato e liquido e, tramite exchange e clearing house, riduce il rischio di default.

\paragraph{Notazione}
Si indica con \(F(0,T)\) il forward price e con \(f(0,T)\) il futures price.  
Il valore di un forward stipulato a \(0\) e osservato a \(t\) è
\[
V_F(t)=\big(F(t,T)-F(0,T)\big)e^{-r(T-t)}.
\]

\paragraph{Hedging con forward e copertura perfetta}
Si considera un portafoglio \(P=(1,N)\) composto da 1 unità dell’asset e \(N\) forward con maturity \(T\), con \(N>0\) long forward e \(N<0\) short forward.  
Il valore del portafoglio è \(V_p(t)=V_S(t)+N V_F(t)\). Poiché \(V_p(0)=S(0)\), alla maturity si ha
\[
V_p(T)=S(T)+N\big(S(T)-F(0,T)\big).
\]
Nel caso \(N=-1\) si ottiene
\[
\Delta V_p(T)=F(0,T)-S(0)e^{rT},
\]
che diventa costante e pari a zero se vale \(F(0,T)=S(0)e^{rT}\). In tale situazione si ottiene \textbf{perfect hedging} e la varianza della variazione di valore è nulla.

\paragraph{Hedging con futures e basis risk}
Si introduce un \textbf{hedge horizon} \(t_h<T\), scelto tipicamente vicino a \(T\).  
Si definisce la \textbf{basis} come \(b(t,T)=S(t)-f(t,T)\). Si ha \(b(T,T)=0\) e \(f(t,T)\to f(T,T)=S(T)\), ma per \(t<t_h<T\) la basis è in generale diversa da zero e può variare nel tempo, generando \textbf{basis risk}.

Si costruisce un portafoglio \(P=(1,N)\) con 1 unità dell’asset e \(N\) futures. Con l’ipotesi operativa \(r=0\), si usa \(V_p(0)=S(0)\) e
\[
V_p(t_h)=S(t_h)+N\big(f(t_h,T)-f(0,T)\big), \quad \Rightarrow \quad
\Delta V_p(t_h)=\big(S(t_h)-S(0)\big)+N\big(f(t_h,T)-f(0,T)\big).
\]
Ponendo \(\Delta S(t_h)=S(t_h)-S(0)\) e \(\Delta f(t_h)=f(t_h,T)-f(0,T)\), si ottiene
\(
\Delta V_p(t_h)=\Delta S(t_h)+N\Delta f(t_h).
\)

\paragraph{Obiettivo e funzione varianza}
Si sceglie \(N\) per minimizzare la varianza della variazione di valore del portafoglio,
\(
\operatorname{Var}\big(\Delta V_p(t_h)\big).
\)
Usando la formula della varianza della somma di due variabili aleatorie,
\[
\operatorname{Var}\big(\Delta V_p(t_h)\big)
=\operatorname{Var}\big(\Delta S(t_h)\big)
+N^2\operatorname{Var}\big(\Delta f(t_h)\big)
+2N\operatorname{Cov}\big(\Delta S(t_h),\Delta f(t_h)\big).
\]
La funzione è convessa in \(N\), quindi il minimo si ottiene imponendo la derivata nulla.

\paragraph{Hedge ratio ottimo}
Derivando rispetto a \(N\) e imponendo lo zero,
\[
\frac{d}{dN}\operatorname{Var}\big(\Delta V_p(t_h)\big)
=2N\operatorname{Var}\big(\Delta f(t_h)\big)
+2\operatorname{Cov}\big(\Delta S(t_h),\Delta f(t_h)\big)=0,
\]
da cui
\[
N^*=-\frac{\operatorname{Cov}\big(\Delta S(t_h),\Delta f(t_h)\big)}
{\operatorname{Var}\big(\Delta f(t_h)\big)}.
\]

\paragraph{Interpretazione economica di \(N^*\)}
La quantità \(N^*\) rappresenta il \textbf{numero ottimo di contratti futures} da includere nel portafoglio per ridurre al minimo la variabilità complessiva della variazione di valore sull’orizzonte \((0,t_h)\).  
Se \(\operatorname{Cov}(\Delta S,\Delta f)>0\), allora \(N^*<0\) e si assume una posizione \textbf{short} nei futures, poiché le variazioni positive dell’asset tendono a essere compensate da variazioni negative della componente futures del portafoglio.
Scrivendo \(\operatorname{Cov}(\Delta S,\Delta f)=\rho\,\sigma_S\sigma_f\), si ottiene anche
\[
N^*=-\rho\,\frac{\sigma_S}{\sigma_f},
\]
che evidenzia come l’hedge ratio dipenda dalla \textbf{correlazione} tra asset e futures e dal rapporto tra le loro deviazioni standard.

\textbf{NB} L'effetto dell'hedging è definito come la differenza relativa tra asset variance e il hedgin portfolio asset.
\emph{
\paragraph{Collegamento con regressione lineare}
La forma di \(N^*\) richiama la regressione lineare
\(
\Delta S=\alpha+\beta \Delta f+\varepsilon,
\)
in cui
\[
\beta=\frac{\operatorname{Cov}(\Delta S,\Delta f)}{\operatorname{Var}(\Delta f)},
\]
quindi \(\beta=-N^*\).  
Il coefficiente di regressione identifica la quantità di futures che minimizza la varianza della variazione di valore del portafoglio sull’intervallo \((0,t_h)\).
}
\emph{
\paragraph{Casi notevoli e unitary hedge}
Se \(N^*=0\), allora \(\beta=0\) e non si riesce a fare hedging in modo efficace.  
Se \(N^*=1\), allora \(\beta=-1\) e si ha \textbf{unitary hedge}.  
Il \textbf{perfect hedge} è un caso particolare di unitary hedge quando \(t_h=T\) e l’underlying del derivato coincide con l’asset, condizione che elimina il basis risk.
}
\emph{
\paragraph{Effectiveness of the hedge}
Si definisce l’\textbf{effectiveness} come la riduzione relativa della varianza passando dall’asset non coperto al portafoglio coperto,
\[
\frac{\sigma_S^2-\sigma_V^2}{\sigma_S^2}.
\]
Usando il valore ottimo \(N^*\), si ottiene che tale quantità coincide con \(\rho^2\) e anche con \(\beta^2\).  
L’efficacia dell’hedge dipende quindi dalla correlazione lineare tra \(\Delta S\) e \(\Delta f\): maggiore è \(|\rho|\), maggiore è la quota di rischio eliminabile, mentre \(|\rho|<1\) implica varianza residua e quindi presenza di basis risk.
}

\vspace{0.5cm}

\textcolor{GreenSoft}{\section{Options, Payoff, Parity and Bounds}}

\paragraph{RECAP}
La lezione introduce le \textbf{opzioni su azioni}, analizzandone payoff e profitto per le diverse posizioni, e sviluppa i principali risultati di \textbf{non arbitraggio}, in particolare la \textbf{put--call parity}, i \textbf{price bounds} per opzioni europee e americane e il teorema di equivalenza tra call americana ed europea in assenza di dividendi.

\paragraph{Tipologie di opzioni}
Si considerano opzioni su azioni, distinguendo
\begin{itemize}[label=-]
\item \textbf{Call option}: diritto di acquistare (buy) l’asset sottostante
\item \textbf{Put option}: diritto di vendere (sell) l’asset sottostante
\end{itemize}

\paragraph{European e American options}
\begin{itemize}[label=-]
\item \textbf{European options}: esercitabili solo a maturity \(T\)
\item \textbf{American options}: esercitabili in qualunque istante \(t \in [0,T]\)
\end{itemize}
\emph{
\paragraph{Assunzioni e notazione}
Si assume inizialmente \(r=0\).  
La notazione utilizzata è
\begin{itemize}[label=-]
\item \(K\) strike price
\item \(S(T)\) prezzo dell’asset a maturity
\item \((x)^+ = \max\{x,0\}\)
\end{itemize}
}
\paragraph{Posizioni long e short}
Per ciascuna opzione si distinguono due posizioni in cui vale \(\text{short} = -\,\text{long}\).
\begin{itemize}[label=-]
\item \textbf{Long}: acquisto (buy) del diritto
\item \textbf{Short}: vendita (sell) del diritto
\end{itemize}

\paragraph{Long call position}
Il \textbf{payoff} di una long call è
\(
(S(T)-K)^+,
\)
mentre il \textbf{profit} è
\(
(S(T)-K)^+ - C,
\)
dove \(C\) è il call price.  
Si ha perdita limitata pari a \(C\) e profitto potenzialmente illimitato.  
Il break-even è \(S(T)=K+C\).

\paragraph{Short call position} 
Il \textbf{payoff} è
\(
-(S(T)-K)^+,
\)
mentre il \textbf{profit} è
\(
C-(S(T)-K)^+.
\)
Il guadagno massimo è limitato a \(C\), mentre la perdita è illimitata per \(S(T)\to +\infty\).

\paragraph{Long put position}
Il \textbf{payoff} di una long put è
\(
(K-S(T))^+
\),
mentre
il \textbf{profit} è
\(
(K-S(T))^+ - P
\),
dove \(P\) è il put price.  
Il break-even è \(S(T)=K-P\).

\paragraph{Short put position}
Il \textbf{payoff} è
\(
-(K-S(T))^+
\),
mentre il \textbf{profit} è
\(
P-(K-S(T))^+.
\)
Il guadagno massimo è \(P\), mentre la perdita cresce al diminuire di \(S(T)\).

\paragraph{Terminologia}
Si introducono le seguenti definizioni
\begin{itemize}[label=-]
\item \textbf{Intrinsic value}: valore dell’opzione se fosse esercitata immediatamente
\item \textbf{Time value}: differenza tra market value e intrinsic value
\end{itemize}
Inoltre
\begin{itemize}[label=-]
\item \textbf{At the money}: cashflow nullo
\item \textbf{In the money}: cashflow positivo
\item \textbf{Out of the money}: cashflow negativo
\end{itemize}

\paragraph{TEO. Put--call parity per European options}
Per opzioni europee su asset che non paga dividendi vale
\[
V_F(0)=C - P = S(0) - K e^{-rT}
\]
La relazione è ottenuta confrontando il payoff di un portafoglio long call e short put con quello di una long forward position e rapprsenta il valore di una foreward $F$ con delivey price $k \ne F(0,T)$.

\paragraph{Put--call parity con dividendi}
Se l’asset paga dividendi con valore attuale \(\text{div}(0)\), la parità diventa
\[
C - P = S(0) - \text{div}(0) - K e^{-rT}
\]
\emph{
\paragraph{Bounds per European call}
Il prezzo di una European call soddisfa
\[
S(0) - K e^{-rT} \le C \le S(0)
\]
I bounds derivano da argomenti di non arbitraggio.}
\emph{
\paragraph{Bounds per European put}
Il prezzo di una European put soddisfa
\[
- S(0) + K e^{-rT} \le P \le K e^{-rT}
\]}

\paragraph{TEO. Bounds e parità per American options}
Per opzioni americane su asset senza dividendi vale
\[
S(0) - K e^{-rT} \ge C^A - P^A \ge S(0) - K
\]
Con dividendi
\[
S(0) - \text{div}(0) - K e^{-rT} \ge C^A - P^A \ge S(0) - \text{div}(0) - K
\]

\paragraph{Relazione tra European e American prices}
Le opzioni europee sono un caso particolare di quelle americane, quindi
\(
C \le C^A, \; P \le P^A
\), inoltre $C \leq S(0)$, $C\geq S(0)-ke^{-rT}$, $P \leq ke^{-rT}$ e $P\geq ke^{-rT}-S(0)$.

\paragraph{TEO. No-dividend American call theorem}
Se l’asset non paga dividendi, il prezzo della call americana coincide con quello della call europea
\[
C^A = C
\]
Ne segue che non è mai ottimale esercitare anticipatamente una American call in assenza di dividendi.

\paragraph{Time value di un'option} Il \textbf{time value of an option} è la differenza tra il suo prezzo a $t$ e il suo payoff 
\begin{itemize}[label=-]
\item \textbf{European}: \textbf{call} $C(t)-(S(t)-K)^+$, \textbf{put} $P(t)-(K-S(t))^+$
\item \textbf{American}: \textbf{call} $C^A(t)-(S(t)-K)^+$, \textbf{put} $P^A(t)-(K-S(t))^+$
\end{itemize}

\vspace{0.5cm}

\textcolor{GreenLeaf}{\section{One-Period Binomial Model, Arbitrage and Risk-Neutral Valuation}}

\paragraph{RECAP}
La lezione introduce il \textbf{modello binomiale a un periodo} come primo modello discreto per la valutazione di asset rischiosi e derivati finanziari.  
Vengono formalizzati i concetti di \textbf{arbitrage} e \textbf{assenza di arbitrage} e si mostra come, in tale contesto, emergano in modo naturale la \textbf{misura risk-neutral}, i \textbf{contingent claims}, i \textbf{replicating portfolios} e la \textbf{completezza del mercato}.  
Nel modello binomiale privo di arbitraggio, ogni derivato ammette un \textbf{prezzo unico}.

\paragraph{Asset e dinamica del modello}
Si considera un mercato con due asset
\begin{itemize}[label=-]
\item Un \textbf{bond risk-free} \(B\)
\item Uno \textbf{stock rischioso} \(S\)
\end{itemize}

La dinamica è a un solo periodo \(t=0,1\)
\[
B(0)=1, \qquad B(1)=1+r
\]
\[
S(0)=S, \qquad 
S(1)=
\begin{cases}
Su & \text{se } p_u\\
Sd & \text{se } p_d=1-p_u
\end{cases} \quad u> d, \; u>1, \; \; d<1, \; p_u+p_d=1
\]
\emph{
È spesso conveniente riscrivere la dinamica dello stock introducendo una variabile aleatoria \(Z\)
\[
S(0)=S,
\qquad
S(1)=S\cdot Z
\]
dove \(Z\) è una \textbf{variabile aleatoria} tale che
\[
Z =
\begin{cases}
u & \text{con probabilità } p_u,\\
d & \text{con probabilità } p_d,
\end{cases}
\qquad
p_u+p_d=1.
\]
I parametri \(u\) e \(d\) sono detti rispettivamente \textbf{up factor} e \textbf{down factor} e si assume
\(
u>1,
\;
d<1.
\)
La variabile aleatoria \(Z\) rappresenta quindi il \textbf{fattore moltiplicativo} che governa l’evoluzione dello stock tra \(t=0\) e \(t=1\).
}

\paragraph{Portafoglio e value process}
Un portafoglio fisso è descritto da \(h=(x,y)\), dove
\begin{itemize}[label=-]
\item \(x\) rappresenta la quantità investita nel bond
\item \(y\) rappresenta la quantità investita nello stock
\end{itemize}

Il \textbf{value process} del portafoglio è
\[
V^h(t)=xB(t)+yS(t), \qquad t=0,1
\]
e in particolare
\[
V^h(0)=x+yS, \qquad 
V^h(1)=x(1+r)+ySZ
\]

\paragraph{PROP. Arbitrage}
Un portafoglio \(h\) è detto un \textbf{arbitrage} se
\(
V^h(0)=0, \; \text{e} \;
V^h(1)>0 \; \text{con probabilità } 1.
\)
Il modello binomiale è \textbf{arbitrage-free} se e solo se
\[
d < 1+r < u
\]
In tal caso, il fattore risk-free \(1+r\) è una \textbf{combinazione convessa} di \(u\) e \(d\).

\paragraph{Probabilità risk-neutral}
Se vale \(d < 1+r < u\), esistono \textbf{uniche} probabilità \(q_u,q_d>0\) tali che
\[
\begin{cases}
1+r = q_u u + q_d d\\
q_u + q_d = 1
\end{cases}
\Rightarrow
\quad
q_u = \frac{(1+r)-d}{u-d},
\qquad
q_d = 1-q_u=\frac{u-(1+r)}{u-d}
\]
\emph{
\paragraph{Misura risk-neutral}
La misura di probabilità \(Q\) definita da
\(
Q(Z=u)=q_u,
\;
Q(Z=d)=q_d
\)
è detta \textbf{risk-neutral measure} (o \textbf{martingale measure}).
Sotto \(Q\) vale
\[
S=\mathbb{E}^Q\!\left[\frac{S(1)}{1+r}\right]
\]
cioè il \textbf{prezzo scontato dello stock è una martingala}.
}
\paragraph{Risk-neutral valuation formula}
Nel modello binomiale privo di arbitraggio il prezzo dello stock soddisfa
\[
S=\mathbb{E}^Q\!\left[\frac{S(1)}{1+r}\right]
\]
La formula è \textbf{indipendente dalle probabilità oggettive} \(P\). 
Il modello di mercato \((B,S)\) è \textbf{arbitrage-free} se e solo se esiste una \textbf{martingale measure} \(Q\).
\emph{
\paragraph{Contingent claims}
Un \textbf{contingent claim} è una variabile aleatoria della forma
\(
X=\phi(Z)
\)
equivalentemente
\(
X=\phi(S(1))
\)
e rappresenta il payoff di un contratto al tempo \(t=1\).}
\emph{
\paragraph{Replicating portfolio e raggiungibilità}
Un claim \(X\) è detto \textbf{raggiungibile} se esiste un portafoglio \(h=(x,y)\) tale che
\(
V^h(1)=X \quad \text{con probabilità } 1
\).
In tal caso \(h\) è detto \textbf{replicating portfolio}.
}
\emph{
\paragraph{Completezza del mercato}
Un mercato è detto \textbf{completo} se ogni contingent claim è raggiungibile.  
Nel modello binomiale a un periodo, \textbf{assenza di arbitraggio implica completezza}.
}
\paragraph{PROP. Pricing principle 1}
Se un claim \(X\) è replicabile da un portafoglio \(h\), allora il suo \textbf{unico prezzo privo di arbitraggio} è
\[
\Pi(t,X)=V^h(t), \qquad t=0,1
\]
in particolare
\[
\Pi(0,X)=V^h(0)
\]
\emph{
\paragraph{Pricing tramite sistema lineare}
Dato \(X=\phi(Z)\), il portafoglio replicante \(h=(x,y)\) risolve il sistema
\[
\begin{cases}
x(1+r)+ySu = \phi(u)\\
x(1+r)+ySd = \phi(d)
\end{cases}
\]
che ammette soluzione unica
\[
x=\frac{1}{1+r}\frac{u\phi(d)-d\phi(u)}{u-d},
\qquad
y=\frac{1}{S}\frac{\phi(u)-\phi(d)}{u-d}
\]
}
\emph{
\paragraph{Risk-neutral pricing dei derivati}
Nel modello binomiale arbitrage-free, il prezzo di un contingent claim è
\[
\Pi(0,X)=\frac{1}{1+r}\mathbb{E}^Q[X]
\]
ossia il \textbf{valore atteso scontato del payoff sotto la misura risk-neutral}.
}
\emph{
\paragraph{Call option}
Una call europea con strike \(K\) ha payoff
\(
X=(S(1)-K)^+
\).
Il suo prezzo è
\[
C=\frac{1}{1+r}\Bigl(q_u(Su-K)^+ + q_d(Sd-K)^+\Bigr)
\]
}

\vspace{0.5cm}

\textcolor{OliveSoft}{\section{Multi-Period Binomial Model}}
\textbf{ARRVATA QUA --}
\paragraph{Idea generale della lezione}
Si estende il modello binomiale a più periodi in tempo discreto \(t=0,\dots,T\), con due asset \((B,S)\).  
Si introduce la dinamica ad albero (ricombinante), la rappresentazione tramite variabili indicatrici e la definizione formale di strategia di portafoglio, valore del portafoglio, predicibilità e self-financing.

\paragraph{Modello multi-periodo e dinamica dei due asset}
Si considera un orizzonte fissato \(T\) e un mercato con un bond \(B_t\) e uno stock \(S_t\).  
Si assume un tasso risk-free costante deterministico \(r\) (simple period rate). La dinamica del bond è
\[
B_{n+1}=(1+r)B_n,\qquad B_0=1.
\]
La dinamica dello stock è
\[
S_{n+1}=S_n Z_n,\qquad S_0=s,
\]
dove \(Z_0,\dots,Z_{T-1}\) sono i.i.d. e assumono i due valori \(u\) e \(d\) con probabilità
\[
\mathbb{P}(Z_n=u)=p_u,\qquad \mathbb{P}(Z_n=d)=p_d.
\]

\paragraph{Dinamica ad albero ed esempio \(T=3\)}
Nel caso \(T=3\), gli stati terminali hanno la forma \(S\,u^k d^{3-k}\).  
Si definiscono le variabili indicatrici
\[
X_i=
\begin{cases}
1 & \text{se } Z_i=u,\\
0 & \text{se } Z_i=d,
\end{cases}
\qquad
Y=\sum_{i=1}^{3}X_i,
\]
dove \(Y\) è il numero di up. Allora \(Y\) è binomiale e
\[
\mathbb{P}(Y=k)=\binom{3}{k}p_u^{k}p_d^{3-k}.
\]
Osservazione: l’indipendenza dei passi è cruciale per calcolare correttamente le probabilità.

\paragraph{Proprietà implicita dell’albero}
L’albero è ricombinante nel senso che una mossa ``up'' seguita da una ``down'' dà lo stesso risultato di una ``down'' seguita da una ``up'' (up--down \(\equiv\) down--up).

\paragraph{Definizione: Portfolio strategy e value process}
Una strategia di portafoglio è un processo stocastico \(\{h_t=(x_t,y_t),\ t=1,\dots,T\}\) tale che \(h_t\) è funzione dell’informazione disponibile \((S_0,\dots,S_{t-1})\). Per convenzione si pone \(h_0=h_1\).  
Il value process associato alla strategia \(h\) è
\[
V_t(h)=x_t(1+r)+y_t S_t.
\]
Interpretazione: \(x_t\) è l’ammontare investito nel bond tra \(t-1\) e \(t\), \(y_t\) è il numero di azioni detenute tra \(t-1\) e \(t\). La strategia è contingente (non anticipativa): può dipendere solo dall’informazione passata.

\paragraph{Definizione: Predicibilità}
Un portafoglio è predicibile se \(\forall n=0,1,\dots,T\) il portafoglio \(h_n\) costruito al tempo \(n\) dipende solo dai nodi dell’albero fino al tempo \(n-1\), cioè
\[
h_n=\varphi(S_0,\dots,S_{n-1}).
\]
Osservazione: una strategia di portafoglio è un portafoglio predicibile.

\paragraph{Convenzioni e notazione sul valore}
Per ogni strategia, per convenzione
\[
h_0=h_1,\qquad V^h(0)=x_1+y_1S(0),\qquad V^h(1)=x_1(1+r)+y_1S(1).
\]

\paragraph{Definizione: Self-financing}
Una strategia di investimento è self-financing se il portafoglio costruito al tempo \(n\) per essere mantenuto fino a \(n+1\) è finanziato interamente dalla ricchezza corrente. In particolare,
\[
V^h(n)=V^{h_{n+1}}(n).
\]
Usando \(V^h(n)=x_n(1+r)+y_n S(n)\) e \(V^{h_{n+1}}(n)=x_{n+1}+y_{n+1}S(n)\), la condizione è equivalente alla budget equation
\[
x_n(1+r)+y_n S(n)=x_{n+1}+y_{n+1}S(n).
\]


\textcolor{LimePastel}{\section{Binomial Model, Contingent Claims and American Options}}

\paragraph{Idea generale della lezione}
La lezione estende il \textbf{modello binomiale} a più periodi per lo studio sistematico del pricing di derivati finanziari.  
Vengono introdotti i concetti di \textbf{contingent claim}, \textbf{replicating portfolio}, \textbf{misura risk-neutral} e \textbf{completezza del mercato}.  
Si studia il pricing di opzioni europee tramite \textbf{backward induction} e si introduce la valutazione delle \textbf{opzioni americane}, evidenziando il ruolo dell’esercizio anticipato.

\paragraph{Contingent claims}
Un \textbf{contingent claim} è un contratto finanziario il cui payoff a maturity dipende dal valore dell’asset sottostante.  
Nel modello binomiale multi-periodo, un claim è tipicamente della forma
\[
X = \varphi(S(T)),
\]
dove \(S(T)\) è il prezzo dello stock al tempo finale \(T\).  
L’obiettivo del pricing è determinare il valore del claim ai tempi intermedi e al tempo iniziale.

\paragraph{Misura risk-neutral e martingala}
Nel modello binomiale privo di arbitraggio esiste un’unica \textbf{misura di probabilità risk-neutral} \(Q\).  
Sotto \(Q\), il prezzo scontato dello stock è una martingala:
\[
S(t) = \mathbb{E}^Q\!\left[ \frac{S(t+1)}{1+r} \,\middle|\, \mathcal{F}_t \right].
\]
Le probabilità risk-neutral \(q_u\) e \(q_d\) sono determinate imponendo che il rendimento atteso dello stock coincida con il tasso risk-free.

\paragraph{Pricing risk-neutral}
Il prezzo di un contingent claim replicabile è dato dalla \textbf{valutazione risk-neutral}:
\[
\Pi(t,X) = \mathbb{E}^Q\!\left[ \frac{X}{(1+r)^{T-t}} \,\middle|\, \mathcal{F}_t \right].
\]
Nel modello binomiale multi-periodo, questa valutazione è implementata operativamente tramite \textbf{backward induction} lungo l’albero dei prezzi.

\paragraph{Backward induction}
La valutazione procede partendo dai nodi terminali, dove il valore del claim coincide con il payoff, e risalendo l’albero calcolando, a ogni nodo,
\[
V(t) = \frac{1}{1+r}\bigl(q_u V_u(t+1) + q_d V_d(t+1)\bigr).
\]
Questo metodo permette di ottenere in modo ricorsivo il prezzo del claim a ogni tempo.

\paragraph{Replicating portfolio e completezza}
Un claim è detto \textbf{replicabile} se esiste una strategia self-financing che ne replica esattamente il payoff finale.  
Nel modello binomiale multi-periodo, l’assenza di arbitraggio implica la \textbf{completezza del mercato}: ogni contingent claim è replicabile e ammette un prezzo unico.

\paragraph{Opzioni europee}
Per un’opzione europea, il valore a ogni nodo è dato esclusivamente dal valore di continuazione ottenuto tramite aspettativa risk-neutral scontata.  
Non è possibile l’esercizio anticipato, quindi il pricing coincide con quello del corrispondente contingent claim.

\paragraph{Opzioni americane}
Un’opzione americana può essere esercitata in qualunque istante \(t \le T\).  
Il valore dell’opzione a un nodo intermedio è dato dal massimo tra:
\[
\text{payoff immediato} \quad \text{e} \quad \text{continuation value}.
\]
Formalmente,
\[
V^A(t) = \max\{\varphi(S(t)),\, \text{CV}(t)\}.
\]
Questo criterio introduce una differenza strutturale rispetto alle opzioni europee.

\paragraph{Call americana senza dividendi}
Nel caso di uno stock che non paga dividendi, non è mai ottimale esercitare anticipatamente una call americana.  
Di conseguenza, il prezzo della call americana coincide con quello della call europea:
\[
C^A = C^E.
\]
Questo risultato è una conseguenza diretta del confronto tra payoff immediato e valore di continuazione.

\paragraph{Osservazione conclusiva}
La lezione mostra come il modello binomiale fornisca un quadro completo e coerente per il pricing dei derivati, unificando approcci di arbitraggio, replicazione e valutazione risk-neutral, e costituendo la base concettuale per modelli più avanzati.


\textcolor{YellowSoft}{\section{Delta Hedging, Modello Binomiale e Limite Continuo}}

\paragraph{Obiettivo della lezione}
La lezione introduce il concetto di \textbf{Delta} come misura di sensibilità del prezzo di un’opzione rispetto al prezzo del sottostante e mostra come sfruttarlo per costruire \textbf{portafogli privi di rischio} (Delta-hedging).  
Si sviluppa poi il \textbf{modello binomiale multi-periodo} fino al \textbf{modello di Cox--Ross--Rubinstein}, chiarendo il collegamento con il limite continuo e le ipotesi di mercato efficiente.

\paragraph{Delta di un’opzione}
La \textbf{Delta} di un’opzione è definita come il rapporto tra la variazione del prezzo dell’opzione e la variazione del prezzo del sottostante,
\[
\Delta = \frac{\Delta C}{\Delta S}.
\]
Nel modello binomiale a un periodo, la Delta si calcola come
\[
\Delta = \frac{\phi(S_1(u)) - \phi(S_1(d))}{S_1(u) - S_1(d)},
\]
ed è costante nel nodo considerato.  
Per una call vale \(\Delta>0\), mentre per una put vale \(\Delta<0\).

\paragraph{Delta-hedging e portafoglio riskless}
L’idea centrale è utilizzare la Delta per costruire un \textbf{portafoglio privo di rischio}.  
Si considera una strategia del tipo
\[
h = (\Delta,-1),
\]
ossia una posizione lunga di \(\Delta\) azioni e una posizione corta di una opzione.  
La Delta viene scelta in modo tale che il valore del portafoglio al tempo \(1\) sia uguale in entrambi gli stati,
\[
V_1^h(u)=V_1^h(d),
\]
eliminando così l’incertezza.  
In assenza di arbitraggio, il valore iniziale del portafoglio coincide con il valore attualizzato del payoff certo.

\paragraph{Pricing di un’opzione: tre approcci equivalenti}
Il prezzo di un’opzione può essere ottenuto in tre modi equivalenti:
\begin{itemize}[label=-]
\item valutazione sotto \textbf{misura risk-neutral},
\item costruzione di un \textbf{portafoglio replicante},
\item \textbf{argomento di non-arbitraggio} tramite portafoglio riskless.
\end{itemize}
Nel modello binomiale, tutti e tre conducono allo stesso prezzo.

\paragraph{Modello binomiale multi-periodo}
Si estende il modello a più periodi considerando un albero ricombinante con fattori di crescita \(u\) e \(d\).  
Dopo \(T\) periodi, il prezzo del sottostante è
\[
S_T = S_0 u^k d^{T-k}, \qquad k=0,1,\dots,T,
\]
dove \(k\) è il numero di movimenti up.  
La variabile \(k\) segue una distribuzione binomiale.

\paragraph{Interpretazione temporale e passaggio a intervalli brevi}
Il tempo viene suddiviso in \(n\) intervalli di ampiezza \(h=T/n\).  
All’aumentare di \(n\), il trading avviene sempre più frequentemente e il modello binomiale approssima l’evoluzione continua dei prezzi.

\paragraph{Rendimenti lineari e rendimenti logaritmici}
Si distinguono due tipi di rendimenti:
\[
R_t^{(L)}=\frac{S(t)-S(t-1)}{S(t-1)}, \qquad
R_t^{(\log)}=\ln\!\left(\frac{S(t)}{S(t-1)}\right).
\]
I rendimenti logaritmici sono \textbf{additivi nel tempo}, a differenza dei rendimenti lineari, e risultano più adatti allo studio del limite continuo.

\paragraph{Scelta di \(u\), \(d\) e probabilità}
Nel modello di Cox--Ross--Rubinstein, i parametri vengono scelti in funzione di \(h\) in modo che media e varianza dei log-rendimenti convergano a quelle osservate empiricamente.  
Una possibile scelta è
\[
u=e^{\sigma\sqrt{h}}, \qquad d=e^{-\sigma\sqrt{h}},
\]
con probabilità tali da riprodurre il drift desiderato.

\paragraph{Ipotesi di mercato efficiente}
La costruzione del modello si basa sull’ipotesi di \textbf{mercato efficiente}:  
i prezzi incorporano tutta l’informazione disponibile e le variazioni di prezzo dipendono solo da nuova informazione.  
Di conseguenza, i rendimenti sono indipendenti e il prezzo futuro dipende solo dal prezzo corrente, non dalla storia passata.

\paragraph{Idea chiave da ricordare}
La Delta permette di eliminare il rischio localmente nel modello binomiale.  
Il modello di Cox--Ross--Rubinstein fornisce un ponte naturale tra il mondo discreto e il modello continuo, mantenendo coerenza con assenza di arbitraggio e mercato efficiente.


\textcolor{GoldenSoft}{\section{Wiener Process and Stochastic Calculus}}

\paragraph{Obiettivo della lezione}
La lezione introduce il \textbf{processo di Wiener (o Brownian Motion)} come modello fondamentale per i rendimenti finanziari e come base dello \textbf{stochastic calculus}.  
Si studiano struttura probabilistica, proprietà di dipendenza, filtrazioni, martingale, irregolarità dei cammini e processi browniani con drift.

\paragraph{Processi stocastici}
Un processo stocastico è una collezione di variabili aleatorie indicizzate dal tempo:
\[
X = \{X_t(\omega),\ t \in T,\ \omega \in \Omega\}.
\]
\begin{itemize}[label=-]
\item Tempo continuo: \(t \in T \subseteq \mathbb{R}_+\)
\item Tempo discreto: \(t \in \mathbb{Z}_+\)
\end{itemize}
Per \(t\) fissato, \(X_t\) è una variabile aleatoria; per \(\omega\) fissato, \(X_t(\omega)\) è una \textbf{traiettoria} (sample path).

\paragraph{Incrementi e dipendenza}
L’incremento di un processo è \(X(t)-X(s)\).
\begin{itemize}[label=-]
\item \textbf{Incrementi stazionari}: la distribuzione dipende solo da \(t-s\)
\item \textbf{Incrementi indipendenti}: incrementi su intervalli disgiunti sono indipendenti
\end{itemize}
Queste sono assunzioni tipiche per i rendimenti finanziari.

\paragraph{Filtrazione e informazione}
Una filtrazione \(\{\mathcal{F}_t\}_{t\ge0}\) è una famiglia crescente di \(\sigma\)-algebre:
\[
\mathcal{F}_s \subseteq \mathcal{F}_t \quad \text{per } s \le t.
\]
Rappresenta l’informazione disponibile fino al tempo \(t\).  
Un processo è \textbf{adattato} se \(X_t\) è \(\mathcal{F}_t\)-misurabile.

\paragraph{Proprietà di Markov}
Un processo ha la proprietà di Markov se
\[
\mathbb{P}(X_t \in A \mid \mathcal{F}_s) = \mathbb{P}(X_t \in A \mid X_s), \quad t \ge s.
\]
Il futuro dipende solo dallo stato presente, non dall’intera storia passata.  
Per i prezzi, questa proprietà è coerente con la \textbf{weak form of market efficiency}.

\paragraph{Martingale}
Un processo \(X_t\) è una martingala se
\[
\mathbb{E}[X_t \mid \mathcal{F}_s] = X_s, \quad t \ge s.
\]
Le martingale hanno media costante nel tempo.  
\textbf{Markov e martingala sono concetti distinti}: un processo può essere uno solo dei due.

\paragraph{Processo di Wiener (Brownian Motion)}
Il processo di Wiener \(W(t)\) è definito da:
\begin{itemize}[label=-]
\item \(W(0)=0\)
\item Incrementi indipendenti
\item Incrementi normali: \(W(t)-W(s) \sim \mathcal{N}(0,t-s)\)
\item Traiettorie continue (a.s.)
\end{itemize}
Conseguenze:
\[
W(t) \sim \mathcal{N}(0,t), \quad \mathbb{E}[W(t)]=0,\quad \mathrm{Var}(W(t))=t.
\]
Le distribuzioni finite-dimensionali sono congiuntamente normali e
\[
\mathrm{Cov}(W(t),W(s)) = \min\{t,s\}.
\]

\paragraph{Irregolarità del moto browniano}
I cammini sono continui ma \textbf{non derivabili}.
\begin{itemize}[label=-]
\item Variazione totale infinita
\item Variazione quadratica:
\[
[W,W](t) = t \quad \text{(a.s.)}
\]
\end{itemize}
La variazione quadratica misura la volatilità ed è centrale nello stochastic calculus.

\paragraph{Moto browniano con drift}
Si definisce
\[
B(t) = x_0 + \mu t + \sigma W(t).
\]
Allora
\[
B(t) \sim \mathcal{N}(x_0+\mu t,\ \sigma^2 t).
\]
Su piccoli intervalli:
\[
\Delta B = \mu \Delta t + \sigma \varepsilon \sqrt{\Delta t}, \quad \varepsilon \sim \mathcal{N}(0,1).
\]

\paragraph{Interpretazione finanziaria}
Il moto browniano:
\begin{itemize}[label=-]
\item non è adatto a modellare direttamente i prezzi (può diventare negativo)
\item cattura correttamente il comportamento dei \textbf{rendimenti}
\end{itemize}
Per questo si modellano i log-rendimenti e non i livelli di prezzo.

\paragraph{Conclusione}
Il processo di Wiener è il mattone fondamentale della modellistica continua in finanza.  
Le sue proprietà di incremento, martingala, Markovianità e variazione quadratica permettono lo sviluppo dello stochastic calculus e dei modelli di pricing.


\textcolor{OrangeSoft}{\section{Trading Strategies with Options}}

\paragraph{Idea generale}
Si studiano strategie di trading costruite combinando opzioni europee (call e put) con la stessa maturity.  
L’obiettivo è modellare il payoff finale in funzione di \(S_T\), controllando direzione del rischio, esposizione alla volatilità e profilo rischio–rendimento.

\paragraph{1. Bull spread using calls}
Strategia rialzista costruita con:
\begin{itemize}[label=-]
\item long call con strike \(K_1\),
\item short call con strike \(K_2>K_1\).
\end{itemize}

\[
\text{Profit} = \text{Payoff} - (C_1 - C_2)
\]

Il profitto massimo è limitato, così come la perdita massima.

\paragraph{2. Bull spread using calls – esempio numerico}
Esempio con:
\[
K_1=30,\; C_1=3 \qquad K_2=35,\; C_2=1
\]
Il costo iniziale è \(C_1-C_2=2\).  
Il profitto massimo si ottiene per \(S_T \ge K_2\).

\paragraph{3. Bull spread using puts}
Strategia rialzista costruita con:
\begin{itemize}[label=-]
\item long put con strike \(K_1\),
\item short put con strike \(K_2>K_1\).
\end{itemize}

\[
\text{Profit} = \text{Payoff} - (p_1 - p_2)
\]

Il flusso di cassa iniziale può essere positivo.

\paragraph{4. Bear spread using puts}
Strategia ribassista costruita con:
\begin{itemize}[label=-]
\item short put con strike \(K_1\),
\item long put con strike \(K_2>K_1\).
\end{itemize}

\[
\text{Profit} = \text{Payoff} + p_1 - p_2
\]

Il profitto è limitato e si realizza per prezzi bassi del sottostante.

\paragraph{5. Bear spread using puts – esempio numerico}
Esempio con:
\[
K_1=30,\; p_1=1 \qquad K_2=35,\; p_2=3
\]

Il profitto massimo si ottiene per \(S_T \le K_1\).

\paragraph{6. Butterfly spread using calls}
Strategia che beneficia di \textbf{bassa volatilità}.  
Costruzione:
\begin{itemize}[label=-]
\item long call \(K_1\),
\item short due call \(K_2=\frac{K_1+K_3}{2}\),
\item long call \(K_3\).
\end{itemize}

\[
\text{Profit} = \text{Payoff} + 2C_2 - C_1 - C_3
\]

Il profitto massimo si ottiene per \(S_T = K_2\).

\paragraph{7. Butterfly spread using calls – esempio numerico}
Esempio con:
\[
S_0=61,\; K_1=55,\; C_1=10,\;
K_2=60,\; C_2=7,\;
K_3=65,\; C_3=5
\]

Costo iniziale:
\[
C_1 + C_3 - 2C_2 = 1
\]

\paragraph{8. Butterfly spread using puts}
Costruzione analoga usando put:
\begin{itemize}[label=-]
\item long put \(K_1\),
\item short due put \(K_2\),
\item long put \(K_3\).
\end{itemize}

\[
\text{Profit} = \text{Payoff} + 2p_2 - p_1 - p_3
\]

\paragraph{9. Combinations}
Le combinations combinano call e put sullo stesso sottostante.

\paragraph{10. Long straddle}
Strategia che beneficia di \textbf{movimenti ampi} del prezzo.

\begin{itemize}[label=-]
\item long call \(K\),
\item long put \(K\).
\end{itemize}

\[
\text{Profit} = \text{Payoff} - (C + p)
\]

La perdita massima è pari al costo iniziale.

\paragraph{11. Long straddle – esempio numerico}
Esempio con:
\[
K=70,\; T=3/12,\; C=4,\; p=3
\]

Perdita massima:
\[
C+p=7
\]

\paragraph{12. Short straddle}
Strategia opposta al long straddle:
\begin{itemize}[label=-]
\item short call \(K\),
\item short put \(K\).
\end{itemize}

\[
\text{Profit} = \text{Payoff} + (C + p)
\]

Il profitto è limitato, la perdita è potenzialmente illimitata.

\paragraph{13. Strangle}
Strategia simile allo straddle ma con strike diversi:
\begin{itemize}[label=-]
\item long put \(K_1\),
\item long call \(K_2>K_1\).
\end{itemize}

\[
\text{Profit} = \text{Payoff} - (p + C)
\]

Richiede movimenti più ampi rispetto allo straddle.

\paragraph{14. Strangle – esempio numerico}
Esempio con:
\[
K_1=45,\; p=3 \qquad K_2=50,\; C=2
\]

Costo iniziale:
\[
p+C=5
\]

\paragraph{Osservazione finale}
Combinando opzioni con strike diversi è possibile costruire payoff molto flessibili.  
In linea teorica, disponendo di opzioni per ogni strike, qualsiasi payoff funzione di \(S_T\) può essere replicato.


\textcolor{OrangeWarm}{\section{Black--Scholes Model and Stochastic Calculus}}

\paragraph{Obiettivo della lezione}
L’obiettivo è introdurre un modello continuo nel tempo per i prezzi degli asset finanziari, basato su processi stocastici, e sviluppare il calcolo necessario per il pricing risk--neutral dei derivati, in particolare nel modello di Black--Scholes.

\paragraph{Setting di Black--Scholes}
Si considera un mercato continuo nel tempo con:
\begin{itemize}[label=-]
\item un bond privo di rischio \(B(t)\),
\item un’azione \(S(t)\).
\end{itemize}
Il bond evolve secondo
\[
B(t)=e^{rt}B(0),
\]
dove \(r\) è il tasso di interesse costante.

\paragraph{European call option}
Una European call option con maturity \(T\) e strike \(K\) ha payoff
\[
\Phi(S(T))=(S(T)-K)^+.
\]

\paragraph{Proposizione (Risk--neutral pricing in tempo continuo)}
Il prezzo al tempo \(0\) di una European call option è dato da
\[
C(0)=\mathbb{E}^{\mathbb{P}^*}\!\left[e^{-rT}\,(S(T)-K)^+\right],
\]
dove \(\mathbb{P}^*\) è una misura di probabilità tale che il processo scontato \(S(t)e^{-rt}\) sia una martingala.

\paragraph{Log--returns e moto browniano}
Si assume che i log--returns siano normalmente distribuiti, cioè
\[
Y(t)=\ln\!\frac{S(t)}{S(0)} = mt+\sigma W(t),
\]
dove \(W(t)\) è un moto browniano standard. Ne segue che
\[
S(t)=S(0)\exp\!\bigl(mt+\sigma W(t)\bigr).
\]

\paragraph{Osservazione (Condizione di martingala)}
Una condizione necessaria affinché \(S(t)e^{-rt}\) sia una martingala è
\[
\mathbb{E}\!\left[S(t)e^{-rt}\right]=S(0).
\]
Nel modello precedente questa condizione impone
\[
m-r+\frac{\sigma^2}{2}=0.
\]

\paragraph{Cambio di misura e misura risk--neutral}
Si introduce una nuova misura di probabilità \(\mathbb{P}^*\) tale che il processo
\[
W^*(t)=W(t)+\frac{m-r+\sigma^2/2}{\sigma}\,t
\]
sia un moto browniano standard sotto \(\mathbb{P}^*\).  
Questo cambio di misura (giustificato dal teorema di Girsanov) consente di rendere il prezzo scontato un processo martingala.

\paragraph{Conclusione fondamentale}
Sotto la misura risk--neutral \(\mathbb{P}^*\), il processo \(S(t)e^{-rt}\) è una martingala e il pricing dei derivati può essere effettuato tramite aspettativa scontata.

\paragraph{Deduzione esplicita della formula di Black--Scholes}
Nel modello di Black--Scholes il prezzo di una European call option è
\[
C(0)=S(0)N(d_1)-Ke^{-rT}N(d_2),
\]
dove
\[
d_1=\frac{\ln\!\bigl(S(0)/K\bigr)+\left(r+\frac{1}{2}\sigma^2\right)T}{\sigma\sqrt{T}},
\qquad
d_2=d_1-\sigma\sqrt{T},
\]
e \(N(\cdot)\) è la funzione di ripartizione della normale standard.

\paragraph{Osservazione cruciale}
Il prezzo dell’opzione non dipende dal drift \(m\) sotto la misura storica, ma solo dai parametri \(r\) e \(\sigma\).

\paragraph{Motivazione del calcolo stocastico}
Poiché le traiettorie del moto browniano hanno variazione non limitata, non è possibile interpretare \(dW(t)\) in senso classico. È quindi necessario introdurre il concetto di integrale stocastico.

\paragraph{Equazione differenziale stocastica (SDE)}
Un processo \(X(t)\) è descritto dall’equazione
\[
dX(t)=\mu(t,X(t))\,dt+\sigma(t,X(t))\,dW(t),
\qquad X(0)=x_0,
\]
dove \(\mu\) è il termine di drift e \(\sigma\) il termine di diffusione.

\paragraph{Definizione (Spazio \(L^2\))}
Un processo \(g(t)\) appartiene a \(L^2([a,b])\) se:
\begin{itemize}[label=-]
\item \(\displaystyle \int_a^b \mathbb{E}[g^2(s)]\,ds<\infty\),
\item \(g(t)\) è adattato alla filtrazione generata dal moto browniano.
\end{itemize}

\paragraph{Definizione (Integrale stocastico di It\^o)}
Per \(g\in L^2([a,b])\) si definisce l’integrale stocastico
\[
\int_a^b g(s)\,dW(s)
\]
come limite in \(L^2\) di integrali definiti per processi semplici.

\paragraph{Proposizione (Propriet\`a fondamentali dell’integrale di It\^o)}
Sia \(g\in L^2([a,b])\). Allora:
\begin{itemize}[label=-]
\item \(\displaystyle \mathbb{E}\!\left[\int_a^b g(s)\,dW(s)\right]=0\),
\item \(\displaystyle \mathbb{E}\!\left[\left(\int_a^b g(s)\,dW(s)\right)^2\right]
= \int_a^b \mathbb{E}[g^2(s)]\,ds\) (isometria di It\^o),
\item \(\displaystyle \int_a^b g(s)\,dW(s)\) è \(\mathcal{F}_b\)-misurabile.
\end{itemize}

\paragraph{Corollario (Martingala)}
Per ogni \(g\in L^2\), il processo
\[
X(t)=\int_0^t g(s)\,dW(s)
\]
è una martingala rispetto alla filtrazione naturale del moto browniano.

\paragraph{Lemma (Caratterizzazione delle martingale)}
Un processo stocastico \(X(t)\) che ammette una rappresentazione differenziale è una martingala se e solo se la sua equazione differenziale non contiene il termine \(dt\), cioè
\[
dX(t)=g(t)\,dW(t).
\]

\paragraph{Implicazione}
Per un processo di It\^o
\[
dX(t)=\mu(t,X(t))\,dt+\sigma(t,X(t))\,dW(t),
\]
la condizione \(\mu\equiv 0\) è necessaria e sufficiente affinché \(X(t)\) sia una martingala.


\textcolor{RedSoft}{\section{Stochastic Calculus and Itô's Formula}}

\paragraph{Idea generale della lezione}
La lezione introduce gli strumenti fondamentali del \textbf{calcolo stocastico}, con particolare attenzione al \textbf{processo di Wiener}, alla sua \textbf{variazione quadratica} e alla \textbf{formula di Itô}, che rappresenta l’analogo stocastico della regola della catena.  
Tali strumenti sono alla base della modellazione dei prezzi finanziari e del pricing dei derivati nel continuo.

\paragraph{Processo di Wiener}
Un processo \(W(t)\) è detto \textbf{processo di Wiener} se soddisfa:
- \(W(0)=0\);
- incrementi indipendenti e stazionari;
- \(W(t)-W(s)\sim \mathcal{N}(0,t-s)\) per \(t>s\);
- traiettorie continue ma quasi sicuramente non derivabili.

Per un incremento temporale \(\Delta t\), vale:
\[
\mathbb{E}[\Delta W(t)] = 0, 
\qquad 
\mathbb{E}[(\Delta W(t))^2] = \Delta t,
\qquad
\mathrm{Var}((\Delta W(t))^2)=2(\Delta t)^2.
\]

\paragraph{Variazione quadratica}
La proprietà chiave del processo di Wiener è la \textbf{variazione quadratica}:
\[
(dW(t))^2 = dt,
\]
intesa in senso limite.  
Ne segue che
\[
\int_0^t (dW(s))^2 = t,
\]
mentre i termini di ordine superiore in \(dt\) sono trascurabili.

\paragraph{Processi stocastici e SDE}
Un processo stocastico \(X(t)\) è descritto dall’equazione differenziale stocastica
\[
dX(t) = \mu(t,X(t))\,dt + \sigma(t,X(t))\,dW(t),
\qquad X(0)=x_0,
\]
dove \(\mu\) è il termine di \textbf{drift} e \(\sigma\) il termine di \textbf{diffusione}.  
In forma integrale:
\[
X(t)=x_0+\int_0^t \mu(s,X(s))\,ds + \int_0^t \sigma(s,X(s))\,dW(s).
\]

\paragraph{Formula di Itô}
Sia \(X(t)\) soluzione di una SDE e sia \(f(t,x)\in C^{1,2}\).  
Definendo \(Z(t)=f(t,X(t))\), la dinamica di \(Z(t)\) è data dalla \textbf{formula di Itô}:
\[
dZ(t)=
\left(
\frac{\partial f}{\partial t}
+ \mu \frac{\partial f}{\partial x}
+ \frac{1}{2}\sigma^2 \frac{\partial^2 f}{\partial x^2}
\right) dt
+ \sigma \frac{\partial f}{\partial x}\, dW(t).
\]

Il termine \(\frac{1}{2}\sigma^2 \frac{\partial^2 f}{\partial x^2}\) è la principale differenza rispetto al calcolo deterministico ed è dovuto alla variazione quadratica di \(W(t)\).

\paragraph{Moto Browniano con drift}
Se \(\mu(t)=\mu\) e \(\sigma(t)=\sigma\) sono costanti, allora
\[
X(t)=x_0+\mu t+\sigma W(t),
\]
con
\[
\mathbb{E}[X(t)] = x_0+\mu t,
\qquad
\mathrm{Var}(X(t))=\sigma^2 t.
\]
Il processo è detto \textbf{Brownian Motion with drift}.

\paragraph{Processo dei prezzi azionari}
Il prezzo di un’azione non dividend-paying è modellato come
\[
dS(t)=\mu S(t)\,dt + \sigma S(t)\,dW(t).
\]
Applicando la formula di Itô a \(\ln S(t)\), si ottiene
\[
S(t)=S(0)\exp\!\left(\left(\mu-\tfrac{1}{2}\sigma^2\right)t+\sigma W(t)\right),
\]
ovvero un \textbf{Geometric Brownian Motion}.

\paragraph{Dinamica dei forward price}
Il prezzo forward è dato da
\[
F(t,S(t)) = S(t)e^{r(T-t)}.
\]
Usando la formula di Itô:
\[
dF(t)= (\mu-r)F(t)\,dt + \sigma F(t)\,dW(t),
\]
quindi il forward segue anch’esso un moto browniano geometrico.

\paragraph{Modelli a mean reversion}
Un esempio di dinamica per il tasso di interesse è il modello di \textbf{Vasicek (Ornstein–Uhlenbeck)}:
\[
dr(t)=\alpha(\mu-r(t))\,dt+\sigma\,dW(t),
\qquad \alpha>0,
\]
che incorpora la proprietà di \textbf{mean reversion} verso il livello \(\mu\).

\paragraph{Risultati chiave}
- Il processo di Wiener ha variazione quadratica non nulla.
- La formula di Itô è la regola fondamentale del calcolo stocastico.
- I prezzi azionari seguono un moto browniano geometrico.
- Forward price e altri derivati ereditano dinamiche stocastiche tramite Itô.
- La mean reversion è cruciale nella modellazione dei tassi di interesse.


\textcolor{RedMuted}{\section{Black--Scholes--Merton Model and Continuous-Time Pricing}}

\paragraph{Obiettivo della lezione}
La lezione introduce il \textbf{modello di Black--Scholes--Merton} in tempo continuo e ne deriva la formula di pricing per claim contingenti mediante:
assenza di arbitraggio, portafogli auto-finanzianti e valutazione risk-neutral.
Si stabilisce il legame tra \textbf{replicazione}, \textbf{equazione alle derivate parziali} e \textbf{aspettativa sotto misura martingala}.

\paragraph{Setting di mercato}
Si considera un mercato continuo con due asset:
\begin{itemize}[label=-]
\item \textbf{Bond} \(B(t)\) con dinamica deterministica
\[
dB(t)=rB(t)\,dt,
\]
\item \textbf{Stock} \(S(t)\) con dinamica stocastica
\[
dS(t)=\mu S(t)\,dt+\sigma S(t)\,dW(t),
\]
dove \(r,\mu,\sigma\) sono costanti e \(W(t)\) è un moto browniano.
\end{itemize}

\paragraph{Claim contingente}
Un claim europeo è definito come
\[
X=\phi(S(T)),
\]
con valore al tempo \(t\) indicato come
\[
V_X(t)=f(t,S(t)),
\]
dove \(f\) è una funzione sufficientemente regolare.

\paragraph{Portafogli auto-finanzianti in tempo continuo}
Un portafoglio \(h(t)=(h_S(t),h_B(t))\) è \textbf{self-financing} se la variazione del suo valore dipende solo dalle variazioni dei prezzi degli asset:
\[
dV^h(t)=h_S(t)\,dS(t)+h_B(t)\,dB(t).
\]

\paragraph{Proposizione}
Se un portafoglio è auto-finanziante, allora la sua dinamica non dipende dalle variazioni delle strategie \(h(t)\), ma solo dalle variazioni dei prezzi degli asset.

\paragraph{Costruzione del portafoglio replicante}
Si considera un portafoglio costituito da:
\begin{itemize}[label=-]
\item una posizione \(n(t)\) sullo stock,
\item una posizione unitaria sul claim.
\end{itemize}
Il valore del portafoglio è
\[
V^h(t)=n(t)S(t)+f(t,S(t)).
\]

Applicando Itô a \(f(t,S(t))\) e imponendo l’assenza del termine stocastico, si ottiene la condizione di copertura:
\[
n(t)=-\frac{\partial f}{\partial s}(t,S(t)).
\]

\paragraph{Condizione di assenza di arbitraggio}
Un portafoglio privo di rischio deve crescere al tasso risk-free:
\[
dV^h(t)=rV^h(t)\,dt.
\]
Da questa condizione segue che \(f\) deve soddisfare una PDE.

\paragraph{Equazione di Black--Scholes--Merton}
La funzione di prezzo \(f(t,s)\) soddisfa il problema:
\[
\begin{cases}
\displaystyle
\frac{\partial f}{\partial t}
+ r s \frac{\partial f}{\partial s}
+ \frac{1}{2}\sigma^2 s^2 \frac{\partial^2 f}{\partial s^2}
- r f = 0, \\[0.6em]
f(T,s)=\phi(s).
\end{cases}
\]

\paragraph{Teorema di Feynman--Kac}
Sotto opportune condizioni di regolarità, la soluzione della PDE ammette la rappresentazione:
\[
f(t,s)=e^{-r(T-t)}\mathbb{E}^Q\!\left[\phi(S(T))\,\big|\,S(t)=s\right],
\]
dove sotto la misura \(Q\) il processo \(S(t)\) soddisfa
\[
dS(t)=rS(t)\,dt+\sigma S(t)\,dW^Q(t).
\]

\paragraph{Misura risk-neutral}
La misura \(Q\) è tale che il prezzo scontato
\[
S(t)e^{-rt}
\]
è una martingala. Nel modello di Black--Scholes tale misura:
\begin{itemize}[label=-]
\item esiste,
\item è unica,
\item implica la completezza del mercato.
\end{itemize}

\paragraph{Formula di pricing}
Il prezzo di un claim europeo è dato da:
\[
\Pi(t,X)=e^{-r(T-t)}\mathbb{E}^Q\!\left[\phi(S(T))\,\big|\,S(t)\right].
\]

\paragraph{Applicazione: Call europea}
Per il payoff
\[
\phi(S(T))=(S(T)-K)^+,
\]
il prezzo al tempo \(0\) è
\[
C_0=S_0 N(d_1)-Ke^{-rT}N(d_2),
\]
con
\[
d_1=\frac{\ln(S_0/K)+(r+\tfrac12\sigma^2)T}{\sigma\sqrt{T}},
\qquad
d_2=d_1-\sigma\sqrt{T}.
\]

\paragraph{Interpretazione probabilistica}
La quantità \(N(d_2)\) rappresenta la probabilità, sotto la misura risk-neutral \(Q\), che l’opzione call venga esercitata:
\[
N(d_2)=\mathbb{P}^Q(S(T)\ge K).
\]

\paragraph{Conclusione}
La lezione mostra come il pricing in tempo continuo derivi da:
replicazione, assenza di arbitraggio e martingale pricing,
unificando PDE, probabilità e finanza in un unico framework coerente.


\textcolor{RoseSoft}{\section{Black--Scholes, Forward Pricing and Parity Relations}}

\paragraph{Idea generale della lezione}
La lezione estende il modello di Black--Scholes al pricing di opzioni scritte su \emph{forward/futures}.  
L’idea chiave è che il prezzo forward ha una dinamica di tipo geometrico browniano analoga a quella del sottostante, permettendo l’applicazione diretta di una formula di tipo Black--Scholes (detta \emph{Black formula}).  
Si introducono inoltre le relazioni di parità e il principio di linearità del pricing.

\paragraph{Definizione -- Forward price}
Si considera uno stock senza dividendi e un tasso risk--free costante \(r\) in capitalizzazione continua.  
Il forward price con maturity \(T\) è
\[
F(0,T)=S(0)e^{rT}.
\]
Più in generale, per \(t\le T\),
\[
F(t,T)=S(t)e^{r(T-t)}.
\]

\paragraph{Proposizione -- Dinamica del forward}
Sotto la misura storica, il prezzo dello stock segue
\[
\mathrm{d}S(t)=\mu S(t)\,\mathrm{d}t+\sigma S(t)\,\mathrm{d}W(t).
\]
Ne segue che il forward price \(F(t,T)=S(t)e^{r(T-t)}\) soddisfa
\[
\mathrm{d}F(t,T)=(\mu-r)F(t,T)\,\mathrm{d}t+\sigma F(t,T)\,\mathrm{d}W(t).
\]
Il forward ha quindi la stessa volatilità del sottostante.

\paragraph{Idea di pricing}
Poiché il forward price ha dinamica di tipo GBM, il pricing di opzioni su forward può essere ricondotto al caso standard di Black--Scholes, trattando il forward come sottostante.

\paragraph{Teorema -- Call europea su forward (Black formula)}
Si considera una call europea con maturity \(T\) e payoff
\[
(F(T,T)-K)^+.
\]
Indicando con \(F_0\) il forward price osservato al tempo zero, il prezzo della call è
\[
C_0=e^{-rT}\bigl[F_0 N(d_1)-K N(d_2)\bigr],
\]
dove
\[
d_1=\frac{\ln(F_0/K)+\tfrac12\sigma^2 T}{\sigma\sqrt{T}},
\qquad
d_2=d_1-\sigma\sqrt{T}.
\]

\paragraph{Corollario -- Put europea su forward}
Il prezzo della put europea con payoff \((K-F(T))^+\) è
\[
P_0=e^{-rT}\bigl[K N(-d_2)-F_0 N(-d_1)\bigr].
\]

\paragraph{Definizione -- Log--returns e volatilità (modello)}
Fissato un passo temporale \(\Delta t\), si definisce il log--return
\[
Y=\ln\frac{S(t+\Delta t)}{S(t)}.
\]
Nel modello di Black--Scholes vale
\[
Y=(\mu-\tfrac12\sigma^2)\Delta t+\sigma\sqrt{\Delta t}\,Z,
\qquad Z\sim\mathcal{N}(0,1).
\]

\paragraph{Volatilità storica e volatilità implicita}
La volatilità \(\sigma\) può essere stimata a partire da una serie storica di log--returns (volatilità storica) oppure determinata invertendo la formula di prezzo dell’opzione in modo da riprodurre il prezzo di mercato (volatilità implicita).

\paragraph{Proposizione -- Linearità del pricing}
Per payoff \(X\) e \(Y\) e per ogni \(\alpha,\beta\in\mathbb{R}\) vale
\[
\Pi(t,\alpha X+\beta Y)=\alpha\Pi(t,X)+\beta\Pi(t,Y).
\]

\paragraph{Teorema -- Put--call parity su stock}
Per opzioni europee su uno stock non dividend--paying, con stessa maturity \(T\) e strike \(K\), vale
\[
C_0-P_0=S_0-Ke^{-rT}.
\]

\paragraph{Teorema -- Put--call parity su forward}
Per opzioni europee scritte su forward con stessa maturity \(T\) e strike \(K\), vale
\[
C_0-P_0=e^{-rT}(F_0-K).
\]

\paragraph{Stock con dividend yield continuo}
Se lo stock paga un dividend yield continuo \(q\), il forward price è
\[
F(0,T)=S_0e^{(r-q)T}.
\]

\paragraph{Conclusione}
Il modello di Black--Scholes permette di prezzare opzioni su forward/futures tramite la Black formula, verificare coerenza tramite relazioni di parità e costruire payoff complessi mediante combinazioni lineari di strumenti base.




\textcolor{PinkPastel}{\section{The Greeks and Hedging}}

\paragraph{Idea generale della lezione}
La lezione introduce le \textbf{Greeks}, ovvero le derivate parziali del valore di un portafoglio rispetto al prezzo dell’underlying e ai parametri del modello.  
Le Greeks forniscono una misura quantitativa dell’esposizione al rischio e costituiscono la base teorica delle strategie di \textbf{hedging}.

\paragraph{Valore del portafoglio e sensitività}
Sia \(V_p(t,S)\) il valore al tempo \(t\) di un portafoglio basato su un singolo underlying con prezzo \(S\).  
Si è interessati alla sensibilità di \(V_p\) rispetto a:
\begin{itemize}[label=-]
\item variazioni del prezzo dell’underlying;
\item variazioni dei parametri del modello.
\end{itemize}

Formalmente si scrive \(V_p = V_p(t,S,r,\sigma)\).

\paragraph{Definizione (Greeks)}
Le principali Greeks sono definite come:
\[
\Delta_p = \frac{\partial V_p}{\partial S}, \qquad
\Gamma_p = \frac{\partial^2 V_p}{\partial S^2},
\]
\[
\Theta_p = \frac{\partial V_p}{\partial t}, \qquad
\nu_p = \frac{\partial V_p}{\partial \sigma}, \qquad
\rho_p = \frac{\partial V_p}{\partial r}.
\]

Un portafoglio è detto \textbf{neutrale} rispetto a una Greek se la corrispondente derivata è nulla.

\paragraph{Approssimazione di Taylor}
Per piccole variazioni dei parametri, la variazione del valore del portafoglio è approssimata da:
\[
\Delta V_p \approx \Delta_p\,\Delta S + \Theta_p\,\Delta t + \nu_p\,\Delta \sigma + \rho_p\,\Delta r + \frac{1}{2}\Gamma_p (\Delta S)^2.
\]

\paragraph{Proposizione (Greeks della call europea in Black--Scholes)}
Nel modello Black--Scholes, per una call europea con strike \(K\) e maturity \(T\), valgono:
\[
\Delta_c = N(d_1), \qquad
\Gamma_c = \frac{\varphi(d_1)}{S\sigma\sqrt{T-t}},
\]
\[
\Theta_c = -\frac{S\varphi(d_1)\sigma}{2\sqrt{T-t}} - rK e^{-r(T-t)}N(d_2),
\]
\[
\nu_c = S\varphi(d_1)\sqrt{T-t},
\]
dove \(N(\cdot)\) è la funzione di ripartizione della normale standard e \(\varphi(\cdot)\) la sua densità.

\paragraph{Delta hedging}
Si considera un portafoglio con valore \(V_p(t,S)\).  
Per piccole variazioni del prezzo, al primo ordine si ha:
\[
\Delta V_p \approx \Delta_p \Delta S.
\]
Se \(\Delta_p=0\), il portafoglio è insensibile a piccole variazioni di \(S\) ed è detto \textbf{Delta-neutral}.

\paragraph{Costruzione di un portafoglio Delta-neutral}
Sia \(f(t,S)\) il prezzo di un derivato e si consideri una posizione short su \(f\).  
Un portafoglio della forma
\[
V(t,S) = x_S S + y - f(t,S)
\]
è Delta-neutral se:
\[
\Delta_p = x_S - \Delta_f = 0 \quad \Rightarrow \quad x_S = \Delta_f.
\]

\paragraph{Lemma}
Per l’underlying stock vale:
\[
\Delta_S = 1, \qquad \Gamma_S = 0.
\]
Ne segue che lo stock non può essere utilizzato per modificare la Gamma di un portafoglio.

\paragraph{Delta--Gamma hedging}
Per rendere un portafoglio sia Delta- che Gamma-neutral è necessario introdurre almeno due derivati.  
Siano \(F(t,S)\) e \(G(t,S)\) due derivati, e si consideri il portafoglio:
\[
V(t,S) = V_p(t,S) + x_F F(t,S) + x_G G(t,S).
\]

Le condizioni di neutralità sono:
\[
\begin{cases}
\Delta_p + x_F \Delta_F + x_G \Delta_G = 0,\\
\Gamma_p + x_F \Gamma_F + x_G \Gamma_G = 0.
\end{cases}
\]

Il sistema ammette soluzione se i due derivati hanno Gamma diversa da zero.

\paragraph{Schema operativo}
Una procedura efficiente consiste in:
\begin{itemize}[label=-]
\item rendere il portafoglio Gamma-neutral usando un derivato;
\item aggiungere l’underlying per ripristinare la Delta-neutralità.
\end{itemize}

Questo sfrutta il fatto che lo stock ha Gamma nulla.



\textcolor{LilacSoft}{\section{Risk Measures, VaR, ES and Hedging}}

\paragraph{Obiettivo della lezione}
La lezione introduce misure quantitative di rischio per variabili aleatorie che rappresentano perdite, con focus su \emph{Value at Risk} e \emph{Expected Shortfall}, e ne discute l’uso nella gestione del rischio e nelle strategie di hedging tramite strumenti derivati.

\paragraph{Definizione di Loss}
Sia \(V(t)\) il valore di un portafoglio al tempo \(t\).  
Si definisce il profitto come \(V(t+1)-V(t)\) e la perdita (loss) come
\[
L(t+1) = -\bigl(V(t+1)-V(t)\bigr).
\]

\paragraph{Value at Risk (VaR)}
\paragraph{Definizione}
Dato un livello di confidenza \(\alpha\in(0,1)\), il \emph{Value at Risk} di una variabile aleatoria \(L\) è definito come
\[
\mathrm{VaR}_\alpha(L)
:= \inf\{y\in\mathbb{R} : \mathbb{P}(L \le y)\ge \alpha\}
= \inf\{y\in\mathbb{R} : F_L(y)\ge \alpha\}.
\]

\paragraph{Osservazione}
Se \(L\) ha distribuzione continua e \(F_L\) è strettamente crescente, allora
\[
\mathrm{VaR}_\alpha(L) = F_L^{-1}(\alpha).
\]

\paragraph{Proposizione (VaR per una variabile normale)}
Se \(L \sim \mathcal{N}(\mu,\sigma^2)\), allora
\[
\mathrm{VaR}_\alpha(L) = \mu + \sigma\,\Phi^{-1}(\alpha),
\]
dove \(\Phi\) è la funzione di distribuzione cumulativa della \(\mathcal{N}(0,1)\).

\paragraph{Misure di rischio coerenti}
Una misura di rischio \(\rho : \mathcal{M}\to\mathbb{R}\) è detta coerente se soddisfa i seguenti assiomi.

\paragraph{A1. Invarianza per traslazione}
\[
\rho(L+\ell)=\rho(L)+\ell \qquad \forall\,\ell\in\mathbb{R}.
\]

\paragraph{A2. Subadditività}
\[
\rho(L_1+L_2)\le \rho(L_1)+\rho(L_2).
\]

\paragraph{A3. Omogeneità positiva}
\[
\rho(\lambda L)=\lambda \rho(L) \qquad \forall\,\lambda\ge 0.
\]

\paragraph{A4. Monotonicità}
Se \(L_1\le L_2\) quasi sicuramente, allora
\[
\rho(L_1)\le \rho(L_2).
\]

\paragraph{Osservazione}
Il \(\mathrm{VaR}\) non è una misura di rischio coerente poiché, in generale, non soddisfa la subadditività.

\paragraph{Expected Shortfall (ES)}
\paragraph{Definizione}
Dato \(\alpha\in(0,1)\), l’\emph{Expected Shortfall} (o Conditional Value at Risk) è definito come
\[
\mathrm{ES}_\alpha(L)
:= \frac{1}{1-\alpha}\int_\alpha^1 \mathrm{VaR}_u(L)\,du
= \mathbb{E}\bigl[L \mid L > \mathrm{VaR}_\alpha(L)\bigr].
\]

\paragraph{Osservazione}
L’Expected Shortfall misura la perdita media condizionata al superamento del VaR ed è una misura di rischio coerente.

\paragraph{Proposizione (ES per una variabile normale)}
Se \(L \sim \mathcal{N}(\mu,\sigma^2)\), allora
\[
\mathrm{ES}_\alpha(L)
= \mu + \sigma\,\frac{\varphi\!\left(\Phi^{-1}(\alpha)\right)}{1-\alpha},
\]
dove \(\varphi\) è la densità della \(\mathcal{N}(0,1)\).

\paragraph{Portafogli e approssimazione lineare}
Sia \(V(t)=\sum_{i=1}^n h_i S_i(t)\) il valore di un portafoglio.  
Indicando con \(R_i(t)\) i rendimenti dei singoli asset e con \(w_i=\frac{h_i S_i(t)}{V(t)}\) i pesi, il rendimento lineare del portafoglio è
\[
R_V(t)=\sum_{i=1}^n w_i R_i(t).
\]
La perdita è approssimata da
\[
L(t+1)\approx -V(t)\,R_V(t).
\]

\paragraph{Hedging e fattori di rischio}
Il valore di un portafoglio o di un derivato dipende da un insieme di fattori di rischio \(z=(z_1,\dots,z_m)\).  
Per piccole variazioni,
\[
\Delta V \approx \frac{\partial V}{\partial t}\Delta t
+ \sum_{i=1}^m \frac{\partial V}{\partial z_i}\Delta z_i,
\]
che giustifica strategie di hedging basate sulle sensitività (Delta-hedging).

\paragraph{Confronto tra strategie di copertura}
\begin{itemize}[label=-]
\item \emph{Forward}: elimina il rischio fissando il tasso futuro, ma rinuncia a qualsiasi beneficio da movimenti favorevoli.
\item \emph{Opzioni}: consentono una copertura con protezione dal downside e mantenimento dell’upside, al costo del premio.
\end{itemize}

\paragraph{Conclusione}
Il confronto tra VaR ed ES evidenzia i limiti del VaR come misura di rischio e motiva l’uso dell’Expected Shortfall nelle decisioni di gestione del rischio e nelle strategie di hedging con derivati.


\textcolor{PurpleSoft}{\section{Speculating with Derivatives}}

\paragraph{Idea generale}
Si usano i derivati (in particolare le opzioni) come \emph{building blocks} per progettare strumenti/portafogli con un \textbf{payoff prescritto}, coerente con le aspettative dell’investitore sull’andamento futuro del sottostante e con la sua propensione al rischio.

\paragraph{Opzioni come strumenti di costruzione}
Se l’investitore si attende un rialzo del prezzo del titolo e vuole scommettere su tale evento, una scelta naturale è l’acquisto di una call. Poiché una call con strike \(K\) vicino al prezzo corrente è tipicamente più economica del titolo, essa può generare un’esposizione \emph{leveraged} e quindi più rischiosa. Il premio può essere ridotto combinando posizioni long e short su opzioni con strike diversi, ottenendo payoff a tratti lineari.

\paragraph{Bull spread con call}
Si considerano due strike \(K_1<K_2\). Un \emph{bull spread} si ottiene con
\[
\text{Long Call}(K_1) \;+\; \text{Short Call}(K_2).
\]
Il payoff a maturity \(T\) è
\[
\Pi_{\text{bull}}(S_T)= (S_T-K_1)^+-(S_T-K_2)^+,
\]
quindi cresce per \(K_1<S_T<K_2\) e si \emph{satura} per \(S_T\ge K_2\). È adatto quando si prevede un aumento moderato del prezzo.

\paragraph{Bear spread con put}
Si considerano due strike \(K_1<K_2\). Un \emph{bear spread} si costruisce con put a strike diversi, con payoff positivo per valori bassi del sottostante
\[
\text{Short Put}(K_1)\;+\;\text{Long Put}(K_2),
\]
e payoff
\[
\Pi_{\text{bear}}(S_T)=-(K_1-S_T)^+ + (K_2-S_T)^+.
\]
È coerente con l’aspettativa di un calo moderato del prezzo.

\paragraph{Butterfly spread con call}
Si scelgono tre strike \(K_1<K_2<K_3\) e si impone \(K_2=\frac{K_1+K_3}{2}\). La \emph{butterfly} si ottiene con
\[
\text{Long Call}(K_1)\;-\;2\,\text{Short Call}(K_2)\;+\;\text{Long Call}(K_3),
\]
e payoff
\[
\Pi_{\text{butterfly}}(S_T)=(S_T-K_1)^+ -2(S_T-K_2)^+ + (S_T-K_3)^+,
\]
che è massimo in prossimità di \(S_T=K_2\) e tende a \(0\) fuori dall’intervallo \([K_1,K_3]\). È coerente con l’idea che il prezzo rimanga circa stabile (o vari poco).

\paragraph{Decomposizione di payoff target in opzioni}
Si osserva che ogni funzione di payoff \emph{continua} e \emph{a tratti lineare} può essere costruita (o approssimata) come combinazione di call e put con diversi strike. Operativamente si scompone un profilo target in segmenti lineari e si sceglie, per ciascuno strike rilevante, il numero di opzioni in modo da \textbf{replicare le pendenze} dei segmenti.

\paragraph{Schema di case study con modello binomiale}
Si combina la logica di portafoglio con gli strumenti precedenti considerando un investitore con una view sul prezzo futuro \(S_T\) e che accetta rischio per aumentare il rendimento atteso. Si analizzano tre alternative
\[
\text{(i) investimento nel titolo},\qquad
\text{(ii) investimento in call},\qquad
\text{(iii) investimento tramite forward}.
\]
Per modellare la distribuzione di \(S_T\) si usa un \textbf{modello binomiale} su \(N\) passi con probabilità uguali di up e down (\(p=1/2\)), fattori moltiplicativi \(\tilde u,\tilde d\) tali che \(S_{t+1}=S_t\tilde u\) oppure \(S_{t+1}=S_t\tilde d\). Ne segue la rappresentazione
\[
S_T = S_0 \tilde u^{k}\tilde d^{\,N-k},\qquad
\mathbb{P}(k)=\binom{N}{k}\left(\frac12\right)^N.
\]
Il payoff di call è \(\phi(S_T)=(S_T-K)^+\). Il rendimento della call (in forma relativa al premio \(c\)) è espresso come \(\frac{\phi(S_T)-c}{c}\). Per il forward, con forward price \(F\), il payoff è \(S_T-F\), e se è previsto un deposito iniziale \(\text{Dep}\), il rendimento relativo è \(\frac{S_T-F}{\text{Dep}}\).

\paragraph{Loss e Value at Risk (VaR)}
Si considera una variabile aleatoria di perdita \(L\). Si definisce la funzione di ripartizione \(F_L(\ell)=\mathbb{P}(L\le \ell)\). Il \(\mathrm{VaR}_\alpha\) è definito come il più piccolo livello di perdita \(\ell\) tale che la probabilità cumulata raggiunge almeno \(\alpha\)
\[
\mathrm{VaR}_\alpha(L)=\inf\{\ell:\, F_L(\ell)\ge \alpha\}.
\]
Nel caso discreto, con valori \(\{\ell_i\}\) e probabilità \(\{p_i\}\), \(F_L\) è a gradini e \(\mathrm{VaR}_\alpha\) coincide con il primo valore \(\ell\) per cui la somma cumulata delle probabilità dei punti \(\le \ell\) supera (o eguaglia) \(\alpha\).


\end{document}
